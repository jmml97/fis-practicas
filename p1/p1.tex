\documentclass[11pt,a4paper]{article}

\usepackage[headsep=1cm,headheight=3cm,includeheadfoot,left=3.5cm,right=3.5cm,top=1cm,bottom=2cm,a4paper]{geometry}

\linespread{1.3}
\setlength{\parindent}{0pt}
\setlength{\parskip}{1em}

\usepackage[spanish]{babel}

%% Fuentes personalizadas para utilizar con XeTeX
\usepackage{fontspec}
\setmainfont{IBM Plex Sans}
\setmonofont[Scale=0.9]{IBM Plex Mono}

\usepackage{enumitem}
\setlist[itemize]{leftmargin=*}
\setlist[enumerate]{leftmargin=*}

\newcommand{\term}[2]{\textbf{#1}\quad#2\\}
\newcounter{ObjCounter}
\newcommand{\obj}[1]{\addtocounter{ObjCounter}{1}\textbf{\rmfamily OBJ-\theObjCounter}\quad#1\\}

\usepackage{array}

\begin{document}

{\Huge{Fundamentos de Ingeniería de Software}}

\section{Objetivos} % (fold)
\label{sec:estudio_del_dominio_del_problema}

\obj{Prueba}
\obj{Prueba 2}


% section estudio_del_dominio_del_problema (end)

\section{Descripción general del sistema} % (fold)
\label{sec:descripción_general_del_sisetema}

El problema consiste en ofrecer una solución software a una clínica privada para automatizar el sistema de atención al cliente, de fichas médicas y sus instalaciones, así como aportar facilidades al personal administrativo. El desarollo de este software consistirá en un sistema que maneje toda la parte relacionada con los pacientes, médicos, administración e infrastructura.

El sistema mejorará el proceso de gestión de citas, consulta de historial clínico, realización de pagos,... agilizándose los trámites y ofreciendo una gestión más eficaz. Esto permitirá aumentar la productividad tanto del personal sanitario como administrativo.

\obj{El sistema deberá almacenar y gestionar el historial clínico de  los pacientes, así como la información relativa al personal de la clínica.}

\obj{El sistema automatizará la gestión de las citas y la realización de los pagos.}

\obj{La reserva de infraestructuras para el trabajo eficiente de los médicos}

\section{Descripción de los implicados} % (fold)
\label{sec:descripción_de_los_implicados}

	\begin{tabular}{lp{3cm}lp{4cm}}
		Nombre & Descripción & Tipo & Responsabilidad \\
		Paciente & Representa a una persona adscrita al centro & Usuario sistema & Pedir cita. Ver su historial médico. Consultar información sobre la clínica. \\
		Sanitario & Representa al personal médico & Usuario producto & \\
		Administración & Representa a una encargado de administración & Usuario producto & \\
		Jefe de planta & Representa al encargado de una sección & Usuario producto & \\
		Proveedor & Representa a un proveedor & Usuario sistema & Suministrar medicamentos a la clínica
	\end{tabular}


% section descripción_de_los_implicados (end)


% section descripción_general_del_sisetema (end)

\section{Lista estructurada de requisitos} % (fold)
\label{sec:lista_estructurada_de_requisitos}

\subsection{Requisitos funcionales}
En esta sección se definen y describen las características del sistema que son esenciales para cubrir las necesidades de los usuarios. Para presentarlo de forma más clara los hemos clasificado según el implicado.

\subsubsection{Paciente}
    \rf{Pedir una cita}
    \rf{Consultar y modificar sus datos}
    \rf{Elegir médico y centro}
    \rf{Consultar lista de espera}
    \rf{Consultar sus citas, radiografías y/o facturas}
    \rf{Recordatorio automático de cita}
\subsubsection{Médico}
    \rf{Consultar y administrar citas}
    \rf{Consultar y reservar disponibilidad de las instalaciones}
    \rf{Modificar y consultar historial medico de pacientes}
    \rf{Consultar los tratamientos de los pacientes}
    \rf{Acceder a la base de datos de recetas}
    \rf{Recetar a un paciente}
    \rf{Consultar y modificar su horario}
    \rf{Consultar su propia ficha}
    \rf{Consultar y modificar listas de espera    }
    \rf{Consultar ingresos}
    \rf{Consultar y gestionar vacaciones}
    \rf{Consultar el material disponible}
\subsubsection{Cualquier usuario que acceda a la página web}
    \rf{Tratamientos disponibles}
    \rf{Especialidades}
    \rf{Horarios y horarios de visitas}
    \rf{Instalaciones}
    \rf{Médicos}
\subsubsection{Encargado de contabilidad}
    \rf{Consultar facturación}
    \rf{Consultar y gestionar impuestos}
    \rf{Gestionar y consultar facturas}
  - El administrador puede:
    \rf{Consultar y modificar todas las bases de datos del sistema}
    \rf{Modificar instalaciones, médicos, recetas, aparatos...} disponibles
    \rf{Gestionar el personal asi como modificar sus funcionalidades}
    \rf{Consultar estadística y control de uso}

    \subsection{Requisitos no funcionales}
  \rn{Consistencia en los identificadores del sistema (unicidad). Esto incluye pacientes, médicos, instalaciones, facturas...}
  \rn{onsistencia en el sistema de citas (no dar dos citas en el mismo momento para el mismo médico o paciente)}
  \rn{istema de recordatorio automático de cita}
  \rn{estionar el acceso a la información según el tipo de usuario}
  \rn{estionar estadísticas de uso del sistema}
  \rn{oporte multiplataforma y multilenguaje}


  \subsection{Requisitos de información}

  Se incluye información que es necesaria incluir en el sistema.
  \ri{Pacientes}\\
  Información relativa a los pacientes dados de alta en la clínica.
    \textbf{Contenido:} Id\_cliente, DNI, nombre y apellidos, fecha de nacimiento, teléfono, forma de pago, en caso de emergencia avisar a, historial médico, grupo sanguíneo, enfermedades previas, alergias, anatomías patológicas
  - Relativa al médico:
    - Id_medico
    - DNI
    - Nombre y apellidos
    - Fecha de nacimiento
    - Teléfono
    - Especialización
    - Sueldo
    - Vacaciones
  - Relativa al encargado de la contabilidad:
    - Id_contable
    - DNI
    - Nombre y apellidos
    - Fecha de nacimiento
    - Teléfono
    - Horario
    - Vacaciones
    - Sueldo
  - Relativa al administrador:
    - Id_administrador
    - DNI
    - Nombre y apellidos
    - Fecha de nacimiento
    - Teléfono
    - Horario
    - Vacaciones
    - Sueldo
  - Relativa a las instalaciones:
    - Id_instalación
    - Nombre
    - Tipo
    - Instrumental
    - Localización
  - Relativa a las recetas:
    - Id_receta
    - Nombre
    - Descripción
% section lista_estructurada_de_requisitos (end)

\section{Glosario de términos} % (fold)
\label{sec:glosario_de_términos}

\term{Prueba}{Prueba2}
\term{Prueba3}{Prueba4}


% section glosario_de_términos (end)

	
\end{document}
