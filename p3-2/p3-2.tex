\documentclass[11pt,a4paper]{article}

\usepackage[headsep=1cm,headheight=3cm,left=3.5cm,right=3.5cm,top=2.5cm,bottom=2.5cm,a4paper]{geometry}

\linespread{1.3}
\setlength{\parindent}{0pt}
\setlength{\parskip}{1em}

\usepackage[spanish]{babel}
\usepackage[utf8]{inputenc}

%% Fuentes personalizadas para utilizar con XeTeX
\usepackage[sfdefault]{roboto}
\usepackage[scaled=0.9]{DejaVuSansMono}
\usepackage[T1]{fontenc}

\usepackage{enumitem}
\setlist[itemize]{leftmargin=*}
\setlist[enumerate]{leftmargin=*}

\usepackage{changepage}

\newcommand{\term}[2]{\textbf{#1}\quad#2}

\newcounter{ActCounter}
\newcommand{\act}[1]{\addtocounter{ActCounter}{1}\textbf{\sffamily ACT-\theActCounter}\quad#1\\}

\newcounter{CUCounter}
\newcommand{\cu}[1]{\addtocounter{CUCounter}{1}\textbf{\sffamily CU-\theCUCounter}\quad#1\\}

\usepackage{tabularx}
\usepackage{float}
\usepackage{adjustbox}

\newenvironment{itemizenomargins}
    {\begin{minipage}[t]{1\linewidth}\begin{itemize}}
    {\end{itemize}\end{minipage}}

\title{Práctica 3: Modelo conceptual y contratos \large\\ Fundamentos de Ingeniería del Software}
\author{Sofía Almedia Bruno \and José Antonio Álvarez Ocete \and Miguel Lentisco Ballesteros \and Simón López Vico \and José María Martín Luque}

\begin{document}

\maketitle

\section{Contratos}

\begin{table}[H]
\centering
\label{my-label}
\begin{tabularx}{\textwidth}{l|X}
\textbf{Nombre}          & llamarSiguientePaciente(idSanitario) \\
\textbf{Responsabilidad} & Llamar al siguiente paciente del Sanitario identificado por idSanitario \\
\textbf{Tipo}            & ClinicaFIS \\
\textbf{Notas}           &  \\
\textbf{Excepciones}     & 
\begin{itemizenomargins}
\item El sanitario identificado por idSanitario no existe
\end{itemizenomargins}\\
\textbf{Salida}          & idPaciente \\
\textbf{Precondiciones}  &  \\
\textbf{Poscondiciones}  &
\begin{itemizenomargins}
\item Se creó un enlace entre un objeto de tipo Cita enlazada con un valor de estado PENDIENTE y el paciente identificado por idPaciente 
\end{itemizenomargins}\\
\textbf{Autor} & Sofía Almeida Bruno\\
\end{tabularx}
\end{table}

\begin{table}[H]
\centering
\label{my-label}
\begin{tabularx}{\textwidth}{l|X}
\textbf{Nombre}          & terminarConsulta(idPaciente, idSanitario) \\
\textbf{Responsabilidad} & Finalizar la cita realizada por el Sanitario identificado por idSanitario al Paciente identificado por idPaciente\\
\textbf{Tipo}            & ClinicaFIS \\
\textbf{Notas}           &  \\
\textbf{Excepciones}     & 
\begin{itemizenomargins}
\item El paciente identificado por idPaciente no existe
\item El sanitario identificado por idSanitario no existe
\end{itemizenomargins}\\
\textbf{Salida}          &  \\
\textbf{Precondiciones}  &  \\
\textbf{Poscondiciones}  & \begin{itemizenomargins}
\item Se destruyó el enlace entre el Paciente identificado por idPaciente y la Cita a la que también está enlazado el Sanitario identificado por idSanitario
\item Se actualizó el estado de la cita a ATENDIDA
\end{itemizenomargins}\\
\textbf{Autor}           & Sofía Almeida Bruno
\end{tabularx}
\end{table}

\begin{table}[H]
\centering
\label{my-label}
\begin{tabularx}{\textwidth}{l|X}
\textbf{Nombre}          & iniciarConsulta(idPaciente) \\
\textbf{Responsabilidad} & Comenzar la consulta al Paciente identificado por idPaciente\\
\textbf{Tipo}            & ClinicaFIS \\
\textbf{Notas}           &  \\
\textbf{Excepciones}     & 
\begin{itemizenomargins}
\item El Paciente identificado por idPaciente no existe
\end{itemizenomargins}\\
\textbf{Salida}          &  idHC\\
\textbf{Precondiciones}  &  \\
\textbf{Poscondiciones}  &
\begin{itemizenomargins}
\item Se creó un objeto de la clase HistoriaClinica debidamente inicializado e identificado por idHC
\item Se creó un enlace entre el objeto de la clase HistoriaClinica identificado por idHC y el objeto Paciente identificado por idPaciente
\end{itemizenomargins}\\
\textbf{Autor}           & Sofía Almeida Bruno
\end{tabularx}
\end{table}

\begin{table}[H]
\centering
\label{my-label}
\begin{tabularx}{\textwidth}{l|X}
\textbf{Nombre}          & registrarAlergia(idHC, tipoAlergia, textoExplicativo) \\
\textbf{Responsabilidad} & Almacenar un objeto Alergia de tipo tipoAlergia en el objeto HistoriaClinica identificado por idHC con su textoExplicativo correspondiente\\
\textbf{Tipo}            & ClinicaFIS \\
\textbf{Notas}           &  \\
\textbf{Excepciones}     & 
\begin{itemizenomargins}
\item No existe la HistoriaClinica identificada por idHC
\item No extiste el tipo de alergia tipoAlergia
\end{itemizenomargins}\\
\textbf{Salida}          &  \\
\textbf{Precondiciones}  &  \\
\textbf{Poscondiciones}  &
\begin{itemizenomargins}
\item Se creó un objeto de la clase Alergia identificado por al
\item Se inicializó al.textoExplicativo a textoExplicativo
\item Se inicializó al.tipoAlergia a tipoAlergia
\item Se creó un enlace entre el objeto de la clase Alergia identificado por al y el objeto de la clase HistoriaClinica identificado por idHC
\end{itemizenomargins}\\
\textbf{Autor}           & Sofía Almeida Bruno
\end{tabularx}
\end{table}

\begin{table}[H]
\centering
\label{my-label}
\begin{tabularx}{\textwidth}{l|X}
\textbf{Nombre}          & diagnosticar(idHC, codDiagnostico,textoExplicativo) \\
\textbf{Responsabilidad} & Añadir el Diagnostico identificado por codDignostico a la HistoriaClinica identificada por idHC\\
\textbf{Tipo}            & ClinicaFIS \\
\textbf{Notas}           &  \\
\textbf{Excepciones}     & 
\begin{itemizenomargins}
\item La HistoriaClinica identificada por idHC no existe
\end{itemizenomargins}\\
\textbf{Salida}          &  \\
\textbf{Precondiciones}  &  \\
\textbf{Poscondiciones}  &
\begin{itemizenomargins}
\item Se creó un objeto de la clase Diagnostico identificado diag
\item Se añadió a diag codDiagnostico en el atributo correspondiente
\item Se inicializó diag.fechaAnotacion a la fecha actual
\item Se inicializó diag.comentario a textoExplicativo
\item Se creó un enlace entre el diag y el objeto de la clase HistoriaClinica identificado por idHC
\end{itemizenomargins}\\
\textbf{Autor}           & Sofía Almeida Bruno
\end{tabularx}
\end{table}

\begin{table}[H]
\centering
\label{my-label}
\begin{tabularx}{\textwidth}{l|X}
\textbf{Nombre}          & recetar(idHC, idTratamiento, nombreMedicamento, posologia) \\
\textbf{Responsabilidad} & Añadir una receta a la HistoriaClinica identificada por idHC\\
\textbf{Tipo}            & ClinicaFIS \\
\textbf{Notas}           &  \\
\textbf{Excepciones}     & 
\begin{itemizenomargins}
\item La HistoriaClinica identificada por idHC no existe
\item El Tratamiento identificado por idTratamiento no existe
\end{itemizenomargins}\\
\textbf{Salida}          &  \\
\textbf{Precondiciones}  &  \\
\textbf{Poscondiciones}  &
\begin{itemizenomargins}
\item Se creó un objeto de la clase Receta identificado por rec debidamente inicializado
\item Se creó un enlace entre el objeto de la clase Receta identificado por rec y el objeto Tratamiento identificado por idTratamiento
\item Se actualizó idTratamiento.fechaAnotacion a la fecha actual
\item Se creó un enlace entre el objeto de la clase Tratamiento identificado por idTratamiento y el objeto de la clase HistoriaClinica identificado por idHC
\end{itemizenomargins}\\
\textbf{Autor}           & Sofía Almeida Bruno
\end{tabularx}
\end{table}


\begin{table}[H]
	\centering
	\label{my-label}
	\begin{tabularx}{\textwidth}{l|X}
		\textbf{Nombre}          & imponerTratamiento(idHC, codDiagnostico, textoExplicativo)\\
		\textbf{Responsabilidad} & Asignar el tratamiento relacionado con el diagnóstico identificado con codDiagnostico al paciente con el historial clínico identificado por idHC y añadir textoExplicativo a AnotaciónHC.  \\
		\textbf{Tipo}            & ClinicaFIS \\
		\textbf{Notas}           & La variable textoExplicativo tendrá información de como realizar el tratamiento asignado. \\
		\textbf{Excepciones}     & 
		\begin{itemizenomargins}
			\item No existe una historia clínica identificada por idHC.
			\item No existe un diagnóstico identificado por codDiagnostico.
		\end{itemizenomargins} \\
		\textbf{Salida}          &  \\
		\textbf{Precondiciones}  &  \\
		\textbf{Poscondiciones}  &
		\begin{itemizenomargins}
			\item Se creó un objeto de la clase Tratamiento (trat), debidamente inicializado.
			\item Se creó un enlace entre trat y el objeto de la clase Diagnostico identificado por codDiagnostico. 
			\item Se creó un enlace entre trat y el objeto de la clase HistoriaClinica identificado por idHC.
			\item Se creó un objeto de la clase AnotacionHC (anot), con el atributo comentario=textoExplicativo, y debidamente inicializado.
			\item Se creó un enlace entre anot y el objeto de la clase HistoriaClínica identificado por idHC.
		\end{itemizenomargins} \\
		\textbf{Autor}			 & Simón López Vico
	\end{tabularx}
\end{table}


\begin{table}[H]
	\centering
	\label{my-label}
	\begin{tabularx}{\textwidth}{l|X}
		\textbf{Nombre}          & pedirPrueba(idHC, tipoPrueba, comentario) \\
		\textbf{Responsabilidad} & Pedir una prueba tipoPrueba para realizar sobre el paciente con historial clínico idHC y añadir comentario a AnotaciónHC.\\
		\textbf{Tipo}            & ClinicaFIS \\
		\textbf{Notas}           & La variable comentario contendrá información sobre la prueba a realizar. \\
		\textbf{Excepciones}     &
		\begin{itemizenomargins}
			\item No existe una historia clínica identificada por idHC.
			\item No existe una prueba médica de tipo=tipoPrueba.
			\item No existe el tipo de prueba indicada.
		\end{itemizenomargins} \\
		\textbf{Salida}          &  \\
		\textbf{Precondiciones}  &  \\
		\textbf{Poscondiciones}  &
		\begin{itemizenomargins}
			\item Se creó un enlace entre el objeto de la clase HistoriaClinica identificado por idHC y un objeto de la clase PruebaMedica el cual tiene el atributo tipo=tipoPrueba.
			\item Se creó un objeto de la clase AnotacionHC (anot), con el atributo this.comentario=comentario, y debidamente inicializado.
			\item Se creó un enlace entre anot y el objeto de la clase HistoriaClínica identificado por idHC.
		\end{itemizenomargins} \\
		\textbf{Autor}			 & Simón López Vico 
	\end{tabularx}
\end{table}


\begin{table}[H]
	\centering
	\label{my-label}
	\begin{tabularx}{\textwidth}{l|X}
		\textbf{Nombre}          & desviarAEspecialista(idHC, especialidad, comentario)\\
		\textbf{Responsabilidad} & Asignar al paciente con historial clínico identificado por idHC un médico con especialidadMédica=especialidad y añadir comentario a AnotaciónHC.\\
		\textbf{Tipo}            & ClinicaFIS \\
		\textbf{Notas}           & La variable comentario contendrá información de por qué el paciente necesita un desvío a un especialista. \\
		\textbf{Excepciones}     &
		\begin{itemizenomargins}
			\item No existe una historia clínica identificada por idHC.
			\item No existe una instancia de médico con especialidadMédica=especialidad.
			\item No existe la especialidad indicada.
		\end{itemizenomargins} \\
		\textbf{Salida}          &  \\
		\textbf{Precondiciones}  &  \\
		\textbf{Poscondiciones}  &
		\begin{itemizenomargins}
			\item Se creó un enlace entre el objeto de la clase HistoriaClinica identificado por idHC y un objeto de la clase Médico el cual tiene el atributo especialidadMédica=especialidad.
			\item Se creó un objeto de la clase AnotacionHC (anot), con el atributo this.comentario=comentario, y debidamente inicializado.
			\item Se creó un enlace entre anot y el objeto de la clase HistoriaClínica identificado por idHC.
		\end{itemizenomargins} \\
		\textbf{Autor}			 & Simón López Vico
	\end{tabularx}
\end{table}


\begin{table}[H]
	\centering
	\label{my-label}
	\begin{tabularx}{\textwidth}{l|X}
		\textbf{Nombre}          & crearCM(especialidad, idSanitario) \\
		\textbf{Responsabilidad} & Crear una consulta médica con tespecialidad=especialidad y asociada al sanitario identificado con idSanitario. \\
		\textbf{Tipo}            & ClinicaFIS \\
		\textbf{Notas}           & \\
		\textbf{Excepciones}     &
		\begin{itemizenomargins}
			\item No existe un sanitario identificado con idSanitario.
			\item No existe la especialidad indicada.
			\item El sanitario idSanitario no es especialista en especialidad.
		\end{itemizenomargins} \\
		\textbf{Salida}          & idConsulta \\
		\textbf{Precondiciones}  & \\
		\textbf{Poscondiciones}  & 
\begin{itemizenomargins}
			\item Se creó un objeto de la clase Consulta (cons), con idConsulta y tespecialidad=especialidad, debidamente inicializado.
			\item Se creó un enlace entre cons y el objeto de la clase Sanitario identificado mediante idSanitario.			
		\end{itemizenomargins} \\
		\textbf{Autor}			 & Simón López Vico
	\end{tabularx}
\end{table}



\begin{table}[H]
	\centering
	\label{my-label}
	\begin{tabularx}{\textwidth}{l|X}
		\textbf{Nombre}          & modificarCM(idConsulta, especialidad, idSanitario) \\
		\textbf{Responsabilidad} & Modificar la consulta médica idConsulta asociada al sanitario identificado con idSanitario, con tespecialidad=especialidad. \\
		\textbf{Tipo}            & ClinicaFIS \\
		\textbf{Notas}           &  \\
		\textbf{Excepciones}     &
		\begin{itemizenomargins}
			\item No existe una consulta identificada con idConsulta.
			\item La consulta idConsulta no tiene tespecialidad=especialidad.
			\item No existe un sanitario identificado con idSanitario. 
			\item La consulta idConsulta no está asociada a idSanitario.
			\item No existe la especialidad indicada.
			\item El sanitario idSanitario no es especialista en especialidad.
		\end{itemizenomargins} \\
		\textbf{Salida}          &  \\
		\textbf{Precondiciones}  &  \\
		\textbf{Poscondiciones}  & Se modificó el objeto de la clase Consulta identificado por idConsulta. \\
		\textbf{Autor}			 & Simón López Vico
	\end{tabularx}
\end{table}


\begin{table}[H]
	\centering
	\label{my-label}
	\begin{tabularx}{\textwidth}{l|X}
		\textbf{Nombre}          & eliminarCM(idConsulta)\\
		\textbf{Responsabilidad} & Eliminar la consulta médica identificada por idConsulta. \\
		\textbf{Tipo}            & ClinicaFIS \\
		\textbf{Notas}           &  \\
		\textbf{Excepciones}     &
		\begin{itemizenomargins}
			\item No existe una consulta identificada por idConsulta.
		\end{itemizenomargins} \\
		\textbf{Salida}          &  \\
		\textbf{Precondiciones}  &  \\
		\textbf{Poscondiciones}  & 
		\begin{itemizenomargins}
			\item Se destruyó el enlace entre el objeto de la clase consulta identificado por idConsulta y el sanitario asociado.
			\item Se destruyó el objeto de la clase consulta identificado por idConsulta.
		\end{itemizenomargins} \\
		\textbf{Autor}			 & Simón López Vico
	\end{tabularx}
\end{table}

\begin{table}[H]
	\centering
	\label{my-label}
	\begin{tabularx}{\textwidth}{l|X}
		\textbf{Nombre}          & consultarCM(idConsulta)\\
		\textbf{Responsabilidad} & Obtener la información del objeto de la clase Consulta identificado por idConsulta. \\
		\textbf{Tipo}            & ClinicaFIS \\
		\textbf{Notas}           & La información se refiere a la especialidad y al sanitario enlazado con la consulta médica (y cualquier otra información extra que se considere oportuna). \\
		\textbf{Excepciones}     &  
			\begin{itemizenomargins}
				\item No existe un objeto de la clase Consulta identificado por idConsulta.
			\end{itemizenomargins}		\\
		\textbf{Salida}          &  infoConsulta \\
		\textbf{Precondiciones}  &  \\
		\textbf{Poscondiciones}  &  \\
		\textbf{Autor}			 & Miguel Lentisco Ballesteros
	\end{tabularx}
\end{table}

\begin{table}[H]
	\centering
	\label{my-label}
	\begin{tabularx}{\textwidth}{l|X}
		\textbf{Nombre}          & modificarHorarioCM(idConsulta, datosNuevoHorario) \\
		\textbf{Responsabilidad} & Modificar el horario del enlace HorarioUso entre el objeto de clase Consulta identificado por idConsulta y el relacionado de clase Recurso, con datosNuevoHorario. \\
		\textbf{Tipo}            & ClinicaFIS \\
		\textbf{Notas}           &  \\
		\textbf{Excepciones}     &  
			\begin{itemizenomargins}
				\item No existe un objeto de la clase Consulta identificado por idConsulta.
				\item El objeto de clase Consulta identificado por idConsulta no tiene un enlace de la clase de asociación HorarioUso con un objeto de clase Recurso. 
				\item La modificación del horario entra en conflicto con otra asignación ya existente.
			\end{itemizenomargins}		\\
		\textbf{Salida}          &  \\
		\textbf{Precondiciones}  &  \\
		\textbf{Poscondiciones}  &  
			\begin{itemizenomargins}
				\item Se modificaron los atributos, con datosNuevoHorario, del enlace de la clase de asociación HorarioUso entre el objeto Consulta identificado por idConsulta y el objeto Recurso relacionado.
			\end{itemizenomargins} \\
		\textbf{Autor}			 & Miguel Lentisco Ballesteros
	\end{tabularx}
\end{table}

\begin{table}[H]
	\centering
	\label{my-label}
	\begin{tabularx}{\textwidth}{l|X}
		\textbf{Nombre}          & consultarHorarioCM(idConsulta)\\
		\textbf{Responsabilidad} & Obtener el horario asignado del objeto de tipo Consulta identificado por idConsulta. \\
		\textbf{Tipo}            & ClinicaFIS \\
		\textbf{Notas}           &  Si no tiene un horario definido devolver un horario vacío. \\
		\textbf{Excepciones}     &  
			\begin{itemizenomargins}
				\item No existe un objeto de tipo Consulta identificado por idConsulta.
			\end{itemizenomargins}		\\
		\textbf{Salida}          &  infoHorarioConsulta \\
		\textbf{Precondiciones}  &  \\
		\textbf{Poscondiciones}  &  \\
		\textbf{Autor}			 & Miguel Lentisco Ballesteros
	\end{tabularx}
\end{table}


\begin{table}[H]
	\centering
	\label{my-label}
	\begin{tabularx}{\textwidth}{l|X}
		\textbf{Nombre}          & consultaPaciente(idPaciente)\\
		\textbf{Responsabilidad} & Obtener la información del objeto de clase Paciente identificado por idPaciente. \\
		\textbf{Tipo}            & ClinicaFIS \\
		\textbf{Notas}           &  En la información se dará el dni, nombre, numero de tarjeta, dirección, teléfono, fecha de nacimiento y una lista de idCita que tenga. \\
		\textbf{Excepciones}     &  
			\begin{itemizenomargins}
				\item No existe un objeto de tipo Paciente identificado por idPaciente.
			\end{itemizenomargins}		\\
		\textbf{Salida}          &  infoPaciente \\
		\textbf{Precondiciones}  &  
			\begin{itemizenomargins}
				\item El usuario tenía acceso (accesoAutorizado = true).
			\end{itemizenomargins} \\
		\textbf{Poscondiciones}  &  \\
		\textbf{Autor}			 & Miguel Lentisco Ballesteros
	\end{tabularx}
\end{table}


\begin{table}[H]
	\centering
	\label{my-label}
	\begin{tabularx}{\textwidth}{l|X}
		\textbf{Nombre}          & crearPaciente(dni, nombre, numeroTarjeta)\\
		\textbf{Responsabilidad} &  Registrar un nuevo objeto de clase Paciente enlazado con ClinicaFIS, devolviendo la identificación del nuevo objeto con idPaciente\\
		\textbf{Tipo}            & ClinicaFIS \\
		\textbf{Notas}           &  Su historial clínico se creará cuando pase consulta por primera vez. \\
		\textbf{Excepciones}     &  
			\begin{itemizenomargins}
				\item DNI inválido.
				\item Numero de tarjeta inválido.
				\item DNI o número de tarjeta ya registrados en otro objeto de clase Paciente.
			\end{itemizenomargins}		\\
		\textbf{Salida}          &  idPaciente \\
		\textbf{Precondiciones}  &  \\
		\textbf{Poscondiciones}  & 
			\begin{itemizenomargins}
				\item Se creó un nuevo objeto de clase Paciente debidamente inicializado con parámetros dni, nombre y numeroTarjeta y con identificación idPaciente.
				\item Se creó un enlace entre el objeto Paciente identificado por idPaciente y el objeto ClinicaFIS.
			\end{itemizenomargins} \\
		\textbf{Autor}			 & Miguel Lentisco Ballesteros
	\end{tabularx}
\end{table}

\begin{table}[H]
	\centering
	\label{my-label}
	\begin{tabularx}{\textwidth}{l|X}
		\textbf{Nombre}          & consultarEspecialidadesConsultas() \\
		\textbf{Responsabilidad} & Devuelve una lista de las distintas especialidades que hay, mirando el atributo tespecialidad de todos los objetos de tipo Consulta existentes.\\
		\textbf{Tipo}            & ClinicaFIS \\
		\textbf{Notas}           & Si no hubiera objetos de tipo Consulta devolver una lista vacía. \\
		\textbf{Excepciones}     &  \\
		\textbf{Salida}          &  listaEspecialidades \\
		\textbf{Precondiciones}  &  \\
		\textbf{Poscondiciones}  &  \\
		\textbf{Autor}			 & Miguel Lentisco Ballesteros
	\end{tabularx}
\end{table}


\begin{table}[H]
	\centering
	\label{my-label}
	\begin{tabularx}{\textwidth}{l|X}
    \textbf{Nombre}          & verConsultas()\\
		\textbf{Responsabilidad} & Muestra todas las consultas de la Clínica. \\
		\textbf{Tipo}            & ClinicaFIS \\
		\textbf{Notas}           &  \\
		\textbf{Excepciones}     &
		\begin{itemizenomargins}
			\item No hay consultas registradas.
		\end{itemizenomargins} \\
		\textbf{Salida}          &  listadoConsulta\\
		\textbf{Precondiciones}  &  Hay consultas registradas en el sistema.\\
		\textbf{Poscondiciones}  & \\  
		\textbf{Autor}			 & José María Martín Luque
	\end{tabularx}
\end{table}

\begin{table}[H]
	\centering
	\label{pedir-cambio-cita-remoto}
	\begin{tabularx}{\textwidth}{l|X}
    \textbf{Nombre}          & pedirCambioCitaRemoto(idCita)\\
		\textbf{Responsabilidad} & Solicitar un cambio de cita para una cita previamente establecida. \\
		\textbf{Tipo}            & ClinicaFIS \\
		\textbf{Notas}           &  \\
		\textbf{Excepciones}     & 
		\begin{itemizenomargins}
			\item Si la cita identificada por idCita no existe.
		\end{itemizenomargins} \\
		\textbf{Salida}          &  listaPosiblesFechasHoras\\
		\textbf{Precondiciones}  &  Debe existir previamente una cita.\\
		\textbf{Poscondiciones}  & 
		\begin{itemizenomargins}
			\item Se modificó el objeto de la Cita identificado por idCita.
		\end{itemizenomargins} \\
		\textbf{Autor}			 & José María Martín Luque
	\end{tabularx}
\end{table}

\begin{table}[H]
	\centering
	\label{pedir-cambio-cita}
	\begin{tabularx}{\textwidth}{l|X}
    \textbf{Nombre}          & pedirCambioCita(idCita, fecha, hora)\\
		\textbf{Responsabilidad} & Solicitar un cambio de cita para una cita previamente establecida. \\
		\textbf{Tipo}            & ClinicaFIS \\
		\textbf{Notas}           &  \\
		\textbf{Excepciones}     & 
		\begin{itemizenomargins}
			\item Si la cita identificada por idCita no existe.
		\end{itemizenomargins} \\
		\textbf{Salida}          &  infoCita\\
		\textbf{Precondiciones}  &  Debe existir previamente una cita.\\
		\textbf{Poscondiciones}  & 
		\begin{itemizenomargins}
			\item Se modificó el objeto de la Cita identificado por idCita.
		\end{itemizenomargins} \\
		\textbf{Autor}			 & José María Martín Luque
	\end{tabularx}
\end{table}

\begin{table}[H]
	\centering
	\label{anular-cita}
	\begin{tabularx}{\textwidth}{l|X}
    \textbf{Nombre}          & anularCita(idCita)\\
		\textbf{Responsabilidad} & Anula una cita previamente establecida. \\
		\textbf{Tipo}            & ClinicaFIS \\
		\textbf{Notas}           &  \\
		\textbf{Excepciones}     & 
		\begin{itemizenomargins}
			\item Si la cita identificada por idCita no existe.
		\end{itemizenomargins} \\
		\textbf{Salida}          &  mensajeConfirmacion\\
		\textbf{Precondiciones}  &  Debe existir previamente una cita.\\
		\textbf{Poscondiciones}  & 
		\begin{itemizenomargins}
			\item Se destruyó el enlace entre el objeto de la clase Consulta identificado por idCita y la agenda del sanitario asociado.
			\item Se destruyó el objeto de la clase Cita identificado por idCita.
		\end{itemizenomargins} \\
		\textbf{Autor}			 & José María Martín Luque
	\end{tabularx}
\end{table}

\begin{table}[H]
	\centering
	\label{anular-cita}
	\begin{tabularx}{\textwidth}{l|X}
    \textbf{Nombre}          & consultarSanitarios(idPaciente)\\
		\textbf{Responsabilidad} & Permite consultar los sanitarios asociados a un paciente. \\
		\textbf{Tipo}            & ClinicaFIS \\
		\textbf{Notas}           &  \\
		\textbf{Excepciones}     & 
		\begin{itemizenomargins}
			\item Si el paciente identificado por idPaciente no existe.
      \item Si no hay sanitarios asociados al paciente.
		\end{itemizenomargins} \\
		\textbf{Salida}          &  listaSanitarios\\
		\textbf{Precondiciones}  &  Debe existir previamente un paciente.\\
		\textbf{Poscondiciones}  & \\
		\textbf{Autor}			 & José María Martín Luque
	\end{tabularx}
\end{table}


\begin{table}[H]
	\centering
	\label{anular-cita}
	\begin{tabularx}{\textwidth}{l|X}
    \textbf{Nombre}          & valorarConsulta(idPaciente, idCita, puntuacion, comentario)\\
		\textbf{Responsabilidad} & Permite valorar el trato y la calidad de una consulta. \\
		\textbf{Tipo}            & ClinicaFIS \\
		\textbf{Notas}           &  \\
		\textbf{Excepciones}     & 
		\begin{itemizenomargins}
			\item Si el paciente identificado por idPaciente no existe.
      \item Si la cita identificada por idCita no existe.
      \item Si no se añade puntuación ni comentario.
		\end{itemizenomargins} \\
		\textbf{Salida}          &  infoValoracion\\
		\textbf{Precondiciones}  &  Debe existir previamente una cita y un paciente asociado.\\
		\textbf{Poscondiciones}  & 
		\begin{itemizenomargins}
			\item Se registró la valoración en el sistema.
		\end{itemizenomargins} \\
		\textbf{Autor}			 & José María Martín Luque
	\end{tabularx}
\end{table}

\end{document}

