\documentclass[11pt,a4paper]{article}

\usepackage[headsep=1cm,headheight=3cm,left=3.5cm,right=3.5cm,top=2.5cm,bottom=2.5cm,a4paper]{geometry}

\linespread{1.3}
\setlength{\parindent}{0pt}
\setlength{\parskip}{1em}

\usepackage[spanish]{babel}
\usepackage[utf8]{inputenc}

%% Fuentes personalizadas para utilizar con XeTeX
\usepackage[sfdefault]{roboto}
\usepackage[scaled=0.9]{DejaVuSansMono}
\usepackage[T1]{fontenc}

\usepackage{enumitem}
\setlist[itemize]{leftmargin=*}
\setlist[enumerate]{leftmargin=*}

\usepackage{changepage}

\newcounter{ActCounter}
\newcommand{\act}[1]{\addtocounter{ActCounter}{1}\textbf{\sffamily ACT-\theActCounter}\quad#1\\}

\usepackage{tabularx}
\usepackage{float}
\usepackage{adjustbox}

\title{Práctica 2: Modelo de casos de uso \large\\ Fundamentos de Ingeniería del Software}
\author{Sofía Almedia Bruno \and José Antonio Álvarez Ocete \and Miguel Lentisco Ballesteros \and Simón López Vico \and José María Martín Luque}

\begin{document}

\maketitle

\section{Introducción}

En el presente documento se muestra el modelo de Casos de Uso obtenido en el proceso de análisis del sistema para la gestión de un centro médico. El modelo se puede descomponer en dos grandes paquetes que agrupan las funcionalidades básicas del sistema.

\section{Diagramas de casos de uso} % (fold)

\begin{figure}[H]
	\caption{Diagrama de casos de uso de gestión de datos}
	\centering
	\includegraphics{diagramas/gestion_datos}
\end{figure}

\begin{figure}[H]
	\caption{Diagrama de casos de uso de contabilidad}
	\centering
	\includegraphics{diagramas/contabilidad}
\end{figure}

\section{Descripción de los actores}
% PLANTILLA PACIENTE

\begin{table}[H]
\label{my-label}
\begin{tabularx}{\textwidth}{l|Xlllr}
	\textbf{Actor}           & \multicolumn{4}{l}{Paciente} & \act\\ 
	\textbf{Descripción}     & \multicolumn{5}{>{\hsize=2\hsize}X}{Paciente adscrito al centro médico que desea pedir cita, consultar su historial o informarse acerca de la clínica}\\
	\textbf{Características} & \multicolumn{5}{>{\hsize=2\hsize}X}{Todos los clientes de la clínica son pacientes}\\ 
	\textbf{Relaciones}      & \multicolumn{5}{>{\hsize=2\hsize}X}{}\\ 
	\textbf{Referencias}     & \multicolumn{5}{>{\hsize=2\hsize}X}{}\\
	\textbf{Autor}           &  & \textbf{Fecha} &  & \textbf{Versión} & \textbf{}                      \\ 
\end{tabularx}
\end{table}

\vspace{1cm}

\begin{table}[H]
\label{my-label}
\begin{tabularx}{\textwidth}{lXl}
	\textbf{Atributos} &  & \\
	\textbf{Nombre}    & \textbf{Descripción} & \textbf{Tipo} \\ \hline
	Datos personales   & Identifican al paciente (\texttt{Id\_paciente}, DNI, nombre y apellidos, ...)     & \\
	Contacto           & Permiten ponerse en contacto con el paciente o algún familiar (teléfono, en caso de emergencia avisar a ...) & \\  
	Datos médicos      & Información relativa a la salud del paciente (historial médico, grupo sanguíneo, enfermedades previas, alergias, ...)            
\end{tabularx}
\end{table}

\vspace{1cm}

\begin{table}[H]
\begin{tabularx}{\textwidth}{lXX}
	\textbf{Comentarios} &  &  \\ \hline
\end{tabularx}
\end{table}
% FIN PLANTILLA PACIENTE

\vspace{2cm}

% PLANTILLA MÉDICO

\begin{table}[H]
\label{my-label}
\begin{tabularx}{\textwidth}{l|Xlllr}
	\textbf{Actor}           & Médico & & & & \act \\ 
	\textbf{Descripción}     & \multicolumn{5}{>{\hsize=2\hsize}X}{Personal sanitario de la clínica}\\
	\textbf{Características} & \multicolumn{5}{>{\hsize=2\hsize}X}{Puede modificar la información del paciente}\\ 
	\textbf{Relaciones}      & \multicolumn{5}{>{\hsize=2\hsize}X}{}\\ 
	\textbf{Referencias}     & \multicolumn{5}{>{\hsize=2\hsize}X}{}\\ 
	\textbf{Autor}           &  & \textbf{Fecha} & & \textbf{Versión} & \\ 
\end{tabularx}
\end{table}

\vspace{1cm}

\begin{table}[H]
\label{my-label}
\begin{tabularx}{\textwidth}{lXl}
	\textbf{Atributos} &  & \\
	\textbf{Nombre}    & \textbf{Descripción} & \textbf{Tipo} \\ \hline
	Datos personales   &  Identifican al médico (\texttt{Id\_médico}, DNI, nombre y apellidos,...)     & \\
	Datos laborales    & Relativos a su trabajo (horario, sueldo, vacaciones,...) &
\end{tabularx}
\end{table}

\vspace{1cm}

\begin{table}[H]
\begin{tabularx}{\textwidth}{lXX}
	\textbf{Comentarios} &  &  \\ \hline
\end{tabularx}
\end{table}

% FIN PLANTILLA MÉDICO

\vspace{2cm}

% PLANTILLA CONTABLE

\begin{table}[H]
	\label{my-label}
	\begin{tabularx}{\textwidth}{l|Xlllr}
		\textbf{Actor}           & \multicolumn{4}{l}{Contable} & \act\\ 
		\textbf{Descripción}     & \multicolumn{5}{>{\hsize=2\hsize}X}{Encargado de gestionar la facturación y los impuestos referentes a la clínica.}\\
		\textbf{Características} & \multicolumn{5}{>{\hsize=2\hsize}X}{Puede consultar las facturas de todo cliente en el sistema y el sueldo del personal, así como etiquetar como moroso a aquel cliente que debe dinero.}\\ 
		\textbf{Relaciones}      & \multicolumn{5}{>{\hsize=2\hsize}X}{Paciente, personal.}\\ 
		\textbf{Referencias}     & \multicolumn{5}{>{\hsize=2\hsize}X}{Consultar facturación, gestionar impuestos, gestionar facturas.}\\
		\textbf{Autor}           & Simón López & \textbf{Fecha} & 04/04/18 & \textbf{Versión} & \textbf{1.0}                      \\ 
	\end{tabularx}
\end{table}

\vspace{1cm}

\begin{table}[H]
	\label{my-label}
	\begin{tabularx}{\textwidth}{lXl}
		\textbf{Atributos}  &  & \\
		\textbf{Nombre}     & \textbf{Descripción} & \textbf{Tipo} \\ \hline
		\textbf{IdContable} & Nombre o pseudónimo referente al contable, único en el sistema que identificará a este. & Texto \\
		\textbf{Salario}    & Cantidad de dinero que cobra el contable al mes. & Numérico \\
	\end{tabularx}
\end{table}

\vspace{1cm}

\begin{table}[H]
	\begin{tabularx}{\textwidth}{lXX}
		\textbf{Comentarios} &  &  \\ \hline
		Usualmente no habrá más de un contable por clínica.
	\end{tabularx}
\end{table}

%FIN DE LA PLANTILLA CONTABLE

% PLANTILLA USUARIO WEB

\begin{table}[H]
	\label{my-label}
	\begin{tabularx}{\textwidth}{l|Xlllr}
		\textbf{Actor}           & \multicolumn{4}{l}{Usuario web} & \act\\ 
		\textbf{Descripción}     & \multicolumn{5}{>{\hsize=2\hsize}X}{Cualquier persona que acceda a la información disponible en la página web}\\
		\textbf{Características} & \multicolumn{5}{>{\hsize=2\hsize}X}{Puede acceder a diversa información tal como las especialidades, tratamientos, horarios, instalaciones o médicos disponibles.}\\ 
		\textbf{Relaciones}      & \multicolumn{5}{>{\hsize=2\hsize}X}{No tiene.}\\ 
		\textbf{Referencias}     & \multicolumn{5}{>{\hsize=2\hsize}X}{}\\
		\textbf{Autor}           & Miguel Lentisco & \textbf{Fecha} & 04/04/18 & \textbf{Versión} & \textbf{1.0}                      \\ 
	\end{tabularx}
\end{table}

\vspace{1cm}

\begin{table}[H]
	\begin{tabularx}{\textwidth}{lXX}
		\textbf{Comentarios} &  &  \\ \hline
		No se va a guardar información sobre estos usuarios, puede ser cualquier persona que quiera informarse sobre la clínica.
	\end{tabularx}
\end{table}

% FIN DE LA PLANTILLA USUARIO WEB

% PLANTILLA TEMPORIZADOR

\begin{table}[H]
	\label{my-label}
	\begin{tabularx}{\textwidth}{l|Xlllr}
		\textbf{Actor}           & \multicolumn{4}{l}{Temporizador} & \act\\ 
		\textbf{Descripción}     & \multicolumn{5}{>{\hsize=2\hsize}X}{Daemon cuyo objetivo es recordar las citas a los demas usuarios del sistema}\\
		\textbf{Características} & \multicolumn{5}{>{\hsize=2\hsize}X}{Accede a los horarios de los distintos usuarios del sistema de forma periódica y avisa tanto a clientes como a trabajadores de su horario.}\\ 
		\textbf{Relaciones}      & \multicolumn{5}{>{\hsize=2\hsize}X}{No tiene.}\\ 
		\textbf{Referencias}     & \multicolumn{5}{>{\hsize=2\hsize}X}{}\\
		\textbf{Autor}           & José Antonio Álvarez & \textbf{Fecha} & 05/04/18 & \textbf{Versión} & \textbf{1.0}                      \\ 
	\end{tabularx}
\end{table}

\vspace{1cm}

\begin{table}[H]
	\begin{tabularx}{\textwidth}{lXX}
		\textbf{Comentarios} &  &  \\ \hline
		Sobre este actor tampoco se almacenan información debido a sus funcionalidades
	\end{tabularx}
\end{table}

% FIN DE LA PLANTILLA TEMPORIZADOR

\section{Descripción de los casos de uso}

% MODIFICAR

\begin{table}[h]
	\centering
	\begin{tabular}{|l|lllll}
		\cline{1-1} \cline{6-6}
		\textbf{Caso de Uso}   & \textbf{Modificar} &   &  & \multicolumn{1}{l|}{\textbf{\textbf{CU(luego pongo el que le toque jeje)}}}        & \multicolumn{1}{l|}{\textbf{}} \\ \cline{1-1} \cline{6-6} 
		\textbf{Actores}       & Paciente, médico  &                                     &                       &                                       & \textbf{}                      \\ \cline{1-1}
		\textbf{Tipo}          & Secundario, esencial  &                                     &                       &                                       & \textbf{}                      \\ \cline{1-1}
		\textbf{Referencias}   & RF-2, RF-9, RF-13, RF-15, RNF-1, RNF-2 &   & Consultar                       &                                       & \textbf{}                      \\ \cline{1-1}
		\textbf{Precondición}  & Plataforma activa y operativa, usuario y/o médico registrados en el sistema.                          &                      &                & \textbf{}                      \\ \cline{1-1}
		\textbf{Postcondición} & -                     &                                     &                       &                                       &                                \\ \cline{1-1} \cline{3-3} \cline{5-5}
		\textbf{Autor}         & \multicolumn{1}{l|}{Simón López} & \multicolumn{1}{l|}{\textbf{Fecha}} & \multicolumn{1}{l|}{05/04/18} & \multicolumn{1}{l|}{\textbf{Versión}} & \textbf{1.0}                      \\ \cline{1-1} \cline{3-3} \cline{5-5}
	\end{tabular}
\end{table}

\begin{table}[h]
	\centering
	\begin{tabular}{l}
		\hline
		\multicolumn{1}{|l|}{Propósito} \\ \hline
		Dar la posibilidad de modificar los datos sobre un elemento del sistema tras haberlos introducido en su registro.
	\end{tabular}
\end{table}

\begin{table}[h]
	\centering
	\begin{tabular}{l}
		\hline
		\multicolumn{1}{|l|}{Resumen} \\ \hline
		El usuario accede a su ficha médica (o el médico a la de éste) y solicita modificar sus datos, introduciendo los nuevos y aceptando la modificación de éstos. También pueden modificarse las listas de espera y el horario de los médicos de manera similar.
	\end{tabular}
\end{table}

% FIN DE MODIFICAR

% INICIO DE RECETAR

\begin{table}[h]
	\centering
	\begin{tabular}{|l|lllll}
		\cline{1-1} \cline{6-6}
		\textbf{Caso de Uso}   & \textbf{Recetar} &   &  & \multicolumn{1}{l|}{\textbf{\textbf{CU(luego pongo el que le toque jeje)}}}        & \multicolumn{1}{l|}{\textbf{}} \\ \cline{1-1} \cline{6-6} 
		\textbf{Actores}       & Paciente, médico  &                                     &                       &                                       & \textbf{}                      \\ \cline{1-1}
		\textbf{Tipo}          & Primario, esencial  &                                     &                       &                                       & \textbf{}                      \\ \cline{1-1}
		\textbf{Referencias}   & RF-12 &   &                       &                                       & \textbf{}                      \\ \cline{1-1}
		\textbf{Precondición}  & Plataforma activa y operativa, usuario y médico registrados en el sistema.                          &                      &                & \textbf{}                      \\ \cline{1-1}
		\textbf{Postcondición} & Historial médico correctamente modificado.                     &                                     &                       &                                       &                                \\ \cline{1-1} \cline{3-3} \cline{5-5}
		\textbf{Autor}         & \multicolumn{1}{l|}{Miguel Lentisco} & \multicolumn{1}{l|}{\textbf{Fecha}} & \multicolumn{1}{l|}{05/04/18} & \multicolumn{1}{l|}{\textbf{Versión}} & \textbf{1.0}                      \\ \cline{1-1} \cline{3-3} \cline{5-5}
	\end{tabular}
\end{table}

\begin{table}[h]
	\centering
	\begin{tabular}{l}
		\hline
		\multicolumn{1}{|l|}{Propósito} \\ \hline
		El médico receta al paciente las medicinas que necesite (añadidas al historial médico)
	\end{tabular}
\end{table}

\begin{table}[h]
	\centering
	\begin{tabular}{l}
		\hline
		\multicolumn{1}{|l|}{Resumen} \\ \hline
		Seleccionado el paciente al que se le va a recetar, el médico solicita ver la lista de recetas disponibles poniendo un filtro adecuado si lo cree necesario (como por nombre, tipo...). El sistema le devuelve la lista de medicamentos disponibles, y el médico selecciona las recetas oportunas, entonces estas se cargan al historial médico del paciente.
	\end{tabular}
\end{table}

% TODO: CAMBIAR esto, sorry por el formato
\begin{itemize}
	\item Médico: selección del recetario
	\item Médico: uso de filtro de búsqueda
	\item Sistema: devuelve lista de recetas disponibles válidas
	\item Médico: selección de recetas
	\item Sistema: recetas añadidas al historial médico
\end{itemize}


\begin{itemize}
	\item Frecuencia esperada: una por cita.
	\item Importancia: baja
	\item Estado: sin implementar
	\item Rendimiento: alto
	\item Urgencia: baja
	\item Estabilidad: alta
\end{itemize}


% FIN DE RECETAR

% Usar la plantilla general

\section{Diagrama de paquetes}

\begin{figure}[H]
	\caption{Diagrama de paquetes}
	\centering
	\includegraphics{diagramas/paquetes}
\end{figure}
	
	
\end{document}
