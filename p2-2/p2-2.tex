\documentclass[11pt,a4paper]{article}

\usepackage[headsep=1cm,headheight=3cm,left=3.5cm,right=3.5cm,top=2.5cm,bottom=2.5cm,a4paper]{geometry}

\linespread{1.3}
\setlength{\parindent}{0pt}
\setlength{\parskip}{1em}

\usepackage[spanish]{babel}
\usepackage[utf8]{inputenc}

%% Fuentes personalizadas para utilizar con XeTeX
\usepackage[sfdefault]{roboto}
\usepackage[scaled=0.9]{DejaVuSansMono}
\usepackage[T1]{fontenc}

\usepackage{enumitem}
\setlist[itemize]{leftmargin=*}
\setlist[enumerate]{leftmargin=*}

\usepackage{changepage}

\newcommand{\term}[2]{\textbf{#1}\quad#2}

\newcounter{ActCounter}
\newcommand{\act}[1]{\addtocounter{ActCounter}{1}\textbf{\sffamily ACT-\theActCounter}\quad#1\\}

\newcounter{CUCounter}
\newcommand{\cu}[1]{\addtocounter{CUCounter}{1}\textbf{\sffamily CU-\theCUCounter}\quad#1\\}

\usepackage{tabularx}
\usepackage{float}
\usepackage{adjustbox}

\title{Práctica 2: Modelo de casos de uso \large\\ Fundamentos de Ingeniería del Software}
\author{Sofía Almedia Bruno \and José Antonio Álvarez Ocete \and Miguel Lentisco Ballesteros \and Simón López Vico \and José María Martín Luque}

\begin{document}

\maketitle

	
% INICIAR CONSULTA %

\begin{table}[H]
	\begin{tabularx}{\textwidth}{l|Xlllr}
		\textbf{Caso de Uso}   & Iniciar Consulta & & & & \cu \\  
		\textbf{Actores}       & Sanitario, Paciente & & & & \\ 
		\textbf{Tipo}          & Primario, esencial & & & & \\
		\textbf{Referencias}   & RF-9, RF-10 & & Abrir HC & &\\
		%\textbf{Referencias}   & \multicolumn{5}{>{\hsize=\dimexpr\textwidth-\hsize\relax}X}{RF-9}\\
		\textbf{Precondición}  & \multicolumn{5}{>{\hsize=\dimexpr\textwidth-0.85\hsize\relax}X}{Plataforma activa y operativa, existencia de una cita con dicho paciente en dicho momento, existencia de historial clínico del paciente; si no es así, se creará.}\\ 
		\textbf{Postcondición} & \multicolumn{5}{>{\hsize=\dimexpr\textwidth-0.85\hsize\relax}X}{Creación del historial clínico del paciente si éste no se encuentra en el sistema.}\\
		\textbf{Autor}         & Simón López & \textbf{Fecha} & 19/04/18 & \textbf{Versión} & 1.0 \\ 
	\end{tabularx}
	
	\bigskip
	
	\begin{tabularx}{\textwidth}{X}
		\textbf{Propósito}\\ \hline
		Comenzar una consulta con un paciente.
	\end{tabularx}
	
	\bigskip
	
	\begin{tabularx}{\textwidth}{X}
		\textbf{Resumen}\\ \hline
		El Sanitario accede al sistema, consultando el historial clínico del paciente y creándolo si éste no existe. Tras ello, comienza la consulta.
	\end{tabularx}
	
	\bigskip
	
	\begin{tabularx}{\textwidth}{X}
		\textbf{Curso Normal (Básico)}\\ \hline
	\end{tabularx}
	\begin{tabularx}{\textwidth}{cXcX}
		\textbf{1} & El sanitario pregunta sus datos al paciente & & \\
		
		\textbf{2} & El paciente responde con sus datos.         & & \\
		
		\textbf{3} & El sanitario introduce los datos del paciente en el sistema para acceder a su historial clínico. & \textbf{4} & Si no existe su historial clínico, se realiza el caso de uso CrearHC. \\
		
		\textbf{5} & El sanitario accede al historial clínico del paciente mediante AbrirHC & & \\
	\end{tabularx}
	
	\begin{tabularx}{\textwidth}{X}
		\textbf{Cursos Alternos}\\ \hline
	\end{tabularx}
	\begin{tabularx}{\textwidth}{cX}
		\textbf{4a} & Si no existe el historial clínico asociado al DNI del paciente:
		\small
		\begin{itemize}
			\item 4a.1: El sanitario pide al paciente sus datos y solicita su historial clínico a su anterior hospital (si lo tiene).
			\item 4a.2: El sanitario introduce en el sistema la información.
			\item 4a.3: El sistema almacena la información referente al paciente con el DNI de éste como clave principal.
		\end{itemize}\\
	\end{tabularx}
\end{table}

\begin{table}[H]
	\begin{tabularx}{\textwidth}{X}
		\textbf{Otros datos}\\ \hline
	\end{tabularx}
	\begin{tabularx}{\textwidth}{lXlX}
		\textbf{Frecuencia esperada} &  & \textbf{Rendimiento} & \\
		\textbf{Importancia} & Vital & \textbf{Urgencia} & Alta\\
		\textbf{Estado} &  & \textbf{Estabilidad} & \\
	\end{tabularx}
	
	\bigskip
	
	\begin{tabularx}{\textwidth}{X}
		\textbf{Comentarios}\\ \hline
		[...] \\
	\end{tabularx}
\end{table}

\newpage




% ABRIR HC %

\begin{table}[H]
	\begin{tabularx}{\textwidth}{l|Xlllr}
		\textbf{Caso de Uso}   & Abrir HC & & & & \cu \\  
		\textbf{Actores}       & Sanitario, Paciente  & & & & \\ 
		\textbf{Tipo}          & Secundario, esencial & & & & \\
		%\textbf{Referencias}   & RF-9 & & CU-2 &\\
		\textbf{Referencias}   & \multicolumn{5}{>{\hsize=\dimexpr\textwidth-\hsize\relax}X}{RF-9, RF-10}\\
		\textbf{Precondición}  & \multicolumn{5}{>{\hsize=\dimexpr\textwidth-0.85\hsize\relax}X}{ }\\ 
		\textbf{Postcondición} & \multicolumn{5}{>{\hsize=\dimexpr\textwidth-0.85\hsize\relax}X}{ }\\
		\textbf{Autor}         & Simón López & \textbf{Fecha} & 20/04/18 & \textbf{Versión} & 1.0 \\ 
	\end{tabularx}
	
	\bigskip
	
	\begin{tabularx}{\textwidth}{X}
		\textbf{Propósito}\\ \hline
		Obtener información sobre el historial clínico de un paciente.
	\end{tabularx}
	
	\bigskip
	
	\begin{tabularx}{\textwidth}{X}
		\textbf{Resumen}\\ \hline
		El sanitario solicita su DNI al paciente y el sistema le proporciona la información que tiene sobre el mismo.
	\end{tabularx}
	
	\bigskip
	
	\begin{tabularx}{\textwidth}{X}
		\textbf{Curso Normal (Básico)}\\ \hline
	\end{tabularx}
	\begin{tabularx}{\textwidth}{cXcX}
		\textbf{1} & El sanitario solicita el DNI al paciente. & \textbf{2} & El sistema busca el historial clínico sobre dicho paciente. \\
		& & \textbf{3} & Se obtiene el historial clínico y se muestra al sanitario. \\
	\end{tabularx}
	
	\begin{tabularx}{\textwidth}{X}
		\textbf{Cursos Alternos}\\ \hline
	\end{tabularx}
	\begin{tabularx}{\textwidth}{cX}
		\textbf{2a} & Si no existe el historial clínico asociado al DNI del paciente se cancela la operación.\\
	\end{tabularx}
\end{table}

\begin{table}[H]
	\begin{tabularx}{\textwidth}{X}
		\textbf{Otros datos}\\ \hline
	\end{tabularx}
	\begin{tabularx}{\textwidth}{lXlX}
		\textbf{Frecuencia esperada} &  & \textbf{Rendimiento} & \\
		\textbf{Importancia} & Alta & \textbf{Urgencia} & Media \\
		\textbf{Estado} &  & \textbf{Estabilidad} & \\
	\end{tabularx}
	
	\bigskip
	
	\begin{tabularx}{\textwidth}{X}
		\textbf{Comentarios}\\ \hline
		[...] \\
	\end{tabularx}
\end{table}

\newpage




% AÑADIR DATOS CLÍNICOS %

\begin{table}[H]
	\begin{tabularx}{\textwidth}{l|Xlllr}
		\textbf{Caso de Uso}   & Añadir datos clínicos & & & & \cu \\  
		\textbf{Actores}       & Sanitario, Paciente  & & & & \\ 
		\textbf{Tipo}          & Secundario, esencial & & & & \\
		%\textbf{Referencias}   & RF-9 & & CU-2 &\\
		\textbf{Referencias}   & \multicolumn{5}{>{\hsize=\dimexpr\textwidth-\hsize\relax}X}{RF-9}\\
		\textbf{Precondición}  & \multicolumn{5}{>{\hsize=\dimexpr\textwidth-0.85\hsize\relax}X}{Existencia del historial clínico al que se quieren añadir los datos.}\\ 
		\textbf{Postcondición} & \multicolumn{5}{>{\hsize=\dimexpr\textwidth-0.85\hsize\relax}X}{Historial clínico modificado sin problemas ni fallos.}\\
		\textbf{Autor}         & Simón López & \textbf{Fecha} & 20/04/18 & \textbf{Versión} & 1.0 \\ 
	\end{tabularx}
	
	\bigskip
	
	\begin{tabularx}{\textwidth}{X}
		\textbf{Propósito}\\ \hline
		Añadir nuevos datos clínicos al historial de un paciente.
	\end{tabularx}
	
	\bigskip
	
	\begin{tabularx}{\textwidth}{X}
		\textbf{Resumen}\\ \hline
		El sanitario pide el DNI al paciente y solicita el cambio del historial clínico de éste, proporcionando los nuevos datos y realizando los cambios en el sistema.
	\end{tabularx}
	
	\bigskip
	
	\begin{tabularx}{\textwidth}{X}
		\textbf{Curso Normal (Básico)}\\ \hline
	\end{tabularx}
	\begin{tabularx}{\textwidth}{cXcX}
		\textbf{1} & El sanitario solicita añadir datos a un historial clínico.   & & \\
		\textbf{2} & El sanitario introduce los nuevos datos. & \textbf{3} & Se almacenan los nuevos datos. \\
		& & \textbf{4} & Se informa de que el proceso se ha realizado correctamente. \\
	\end{tabularx}
	
	\begin{tabularx}{\textwidth}{X}
		\textbf{Cursos Alternos}\\ \hline
	\end{tabularx}
	\begin{tabularx}{\textwidth}{cX}
		\textbf{1a} & Si no existe el historial clínico asociado al DNI del paciente se cancela la operación.\\
		\textbf{2a} & Si los datos a introducir son incorrectos se informa y se piden los datos correctos. Si no se proporcionan los datos, concluye el proceso. \\
	\end{tabularx}
\end{table}

\begin{table}[H]
	\begin{tabularx}{\textwidth}{X}
		\textbf{Otros datos}\\ \hline
	\end{tabularx}
	\begin{tabularx}{\textwidth}{lXlX}
		\textbf{Frecuencia esperada} &  & \textbf{Rendimiento} & \\
		\textbf{Importancia} & Alta & \textbf{Urgencia} & Media \\
		\textbf{Estado} &  & \textbf{Estabilidad} & \\
	\end{tabularx}
	
	\bigskip
	
	\begin{tabularx}{\textwidth}{X}
		\textbf{Comentarios}\\ \hline
		[...] \\
	\end{tabularx}
\end{table}

\newpage




% EXPLORAR PACIENTE %

\begin{table}[H]
	\begin{tabularx}{\textwidth}{l|Xlllr}
		\textbf{Caso de Uso}   & Explorar paciente & & & & \cu \\  
		\textbf{Actores}       & Sanitario, Paciente & & & & \\ 
		\textbf{Tipo}          & Primario, esencial  & & & & \\
		%\textbf{Referencias}   & RF-9 & & CU-2 &\\
		\textbf{Referencias}   & \multicolumn{5}{>{\hsize=\dimexpr\textwidth-\hsize\relax}X}{ }\\
		\textbf{Precondición}  & \multicolumn{5}{>{\hsize=\dimexpr\textwidth-0.85\hsize\relax}X}{ }\\ 
		\textbf{Postcondición} & \multicolumn{5}{>{\hsize=\dimexpr\textwidth-0.85\hsize\relax}X}{Añadir la información obtenida tras la exploración al historial clínico del paciente. }\\
		\textbf{Autor}         & Simón López & \textbf{Fecha} & 20/04/18 & \textbf{Versión} & 1.0 \\ 
	\end{tabularx}
	
	\bigskip
	
	\begin{tabularx}{\textwidth}{X}
		\textbf{Propósito}\\ \hline
		Comprobar el estado del paciente para realizar un diagnóstico.
	\end{tabularx}
	
	\bigskip
	
	\begin{tabularx}{\textwidth}{X}
		\textbf{Resumen}\\ \hline
		El sanitario realiza distintas pruebas al paciente, determinando el estado de salud de éste y añadiendo o eliminando información de su historial clínico sobre las condiciones de salud en las que se encuentra. \\
	\end{tabularx}
	
	\bigskip
	
	\begin{tabularx}{\textwidth}{X}
		\textbf{Curso Normal (Básico)}\\ \hline
	\end{tabularx}
	\begin{tabularx}{\textwidth}{cXcX}
		\textbf{1} & El sanitario realiza pruebas médicas sobre el paciente. & & \\
		\textbf{2} & El sanitario introduce los datos obtenidos mediante las pruebas en el historial clínico del paciente. & \textbf{3} & Se almacenan los nuevos datos. \\
		& & \textbf{4} & Se informa de que el proceso se ha realizado correctamente. \\
	\end{tabularx}
	
	\begin{tabularx}{\textwidth}{X}
		\textbf{Cursos Alternos}\\ \hline
	\end{tabularx}
	\begin{tabularx}{\textwidth}{cX}
		\textbf{2a} & Si no existe el historial clínico asociado al DNI del paciente, se crea un nuevo historial clínico.\\
		\textbf{2b} & Si los datos a introducir son incorrectos se informa y se piden los datos correctos. Si no se proporcionan los datos, concluye el proceso. \\
	\end{tabularx}
\end{table}

\begin{table}[H]
	\begin{tabularx}{\textwidth}{X}
		\textbf{Otros datos}\\ \hline
	\end{tabularx}
	\begin{tabularx}{\textwidth}{lXlX}
		\textbf{Frecuencia esperada} & Una por consulta & \textbf{Rendimiento} & \\
		\textbf{Importancia} & Alta & \textbf{Urgencia} & \\
		\textbf{Estado} &  & \textbf{Estabilidad} & \\
	\end{tabularx}
	
	\bigskip
	
	\begin{tabularx}{\textwidth}{X}
		\textbf{Comentarios}\\ \hline
		[...] \\
	\end{tabularx}
\end{table}


\newpage




% CURA %

\begin{table}[H]
	\begin{tabularx}{\textwidth}{l|Xlllr}
		\textbf{Caso de Uso}   & Cura & & & & \cu \\  
		\textbf{Actores}       & Sanitario, Paciente & & & & \\ 
		\textbf{Tipo}          & Primario, esencial  & & & & \\
		%\textbf{Referencias}   & RF-9 & & & &\\
		\textbf{Referencias}   & \multicolumn{5}{>{\hsize=\dimexpr\textwidth-\hsize\relax}X}{ RF-10 }\\
		\textbf{Precondición}  & \multicolumn{5}{>{\hsize=\dimexpr\textwidth-0.85\hsize\relax}X}{ }\\ 
		\textbf{Postcondición} & \multicolumn{5}{>{\hsize=\dimexpr\textwidth-0.85\hsize\relax}X}{ }\\
		\textbf{Autor}         & Simón López & \textbf{Fecha} & 20/04/18 & \textbf{Versión} & 1.0 \\ 
	\end{tabularx}
	
	\bigskip
	
	\begin{tabularx}{\textwidth}{X}
		\textbf{Propósito}\\ \hline
		Realizar una cura al paciente para mejorar su estado de salud. \\
	\end{tabularx}
	
	\bigskip
	
	\begin{tabularx}{\textwidth}{X}
		\textbf{Resumen}\\ \hline
		El sanitario realiza un tipo de cura sobre el paciente en función del problema que éste tenga. \\
	\end{tabularx}
	
	\bigskip
	
	\begin{tabularx}{\textwidth}{X}
		\textbf{Curso Normal (Básico)}\\ \hline
	\end{tabularx}
	\begin{tabularx}{\textwidth}{cXcX}
		\textbf{1} & El sanitario cura al paciente & & \\
	\end{tabularx}
	
	\begin{tabularx}{\textwidth}{X}
		\textbf{Cursos Alternos}\\ \hline
	\end{tabularx}
	\begin{tabularx}{\textwidth}{cX}
		& \\
	\end{tabularx}
\end{table}

\begin{table}[H]
	\begin{tabularx}{\textwidth}{X}
		\textbf{Otros datos}\\ \hline
	\end{tabularx}
	\begin{tabularx}{\textwidth}{lXlX}
		\textbf{Frecuencia esperada} &  & \textbf{Rendimiento} & \\
		\textbf{Importancia} & Alta & \textbf{Urgencia} & Alta \\
		\textbf{Estado} &  & \textbf{Estabilidad} & \\
	\end{tabularx}
	
	\bigskip
	
	\begin{tabularx}{\textwidth}{X}
		\textbf{Comentarios}\\ \hline
		Una cura no tiene por qué ser realizada sobre todos los pacientes que acudan a la consulta; la cura se realizará si el paciente muestra signos de que es necesaria. \\
	\end{tabularx}
\end{table}

\newpage




% RECETAR MEDICAMENTO %

\begin{table}[H]
	\begin{tabularx}{\textwidth}{l|Xlllr}
		\textbf{Caso de Uso}   & Recetar medicamento & & & & \cu \\  
		\textbf{Actores}       & Médico, Paciente & & & & \\ 
		\textbf{Tipo}          & Secundario, esencial  & & & & \\
		%\textbf{Referencias}   & RF-9 & & CU-2 &\\
		\textbf{Referencias}   &  \multicolumn{5}{>{\hsize=\dimexpr\textwidth-\hsize\relax}X}{ RF-11, RF-12 }\\
		\textbf{Precondición}  &  \multicolumn{5}{>{\hsize=\dimexpr\textwidth-0.85\hsize\relax}X}{ }\\ 
		\textbf{Postcondición} & \multicolumn{5}{>{\hsize=\dimexpr\textwidth-0.85\hsize\relax}X}{ Emisión de una receta para que el paciente pueda obtener el medicamento. }\\
		\textbf{Autor}         & Simón López & \textbf{Fecha} & 20/04/18 & \textbf{Versión} & 1.0 \\ 
	\end{tabularx}
	
	\bigskip
	
	\begin{tabularx}{\textwidth}{X}
		\textbf{Propósito}\\ \hline
		Proporcionar al paciente un medicamento para mejorar su estado de salud. \\
	\end{tabularx}
	
	\bigskip
	
	\begin{tabularx}{\textwidth}{X}
		\textbf{Resumen}\\ \hline
		El médico genera una receta de un medicamento en función del historial clínico y de las necesidades del paciente. \\
	\end{tabularx}
	
	\bigskip
	
	\begin{tabularx}{\textwidth}{X}
		\textbf{Curso Normal (Básico)}\\ \hline
	\end{tabularx}
	\begin{tabularx}{\textwidth}{cXcX}
		\textbf{1} & El médico decide según su conocimiento el medicamento necesario para tratar al paciente. & & \\
		\textbf{2} & El médico introduce los datos de la receta a emitir. & \textbf{3} & Se almacenan los nuevos datos. \\
		& & \textbf{4} & Se informa de que el proceso se ha realizado correctamente. \\
		\textbf{5} & Se entrega una receta al paciente para que éste pueda adquirir el medicamento. & & \\
	\end{tabularx}
	
	\begin{tabularx}{\textwidth}{X}
		\textbf{Cursos Alternos}\\ \hline
	\end{tabularx}
	\begin{tabularx}{\textwidth}{cX}
		\textbf{2a} & Si los datos introducidos son incorrectos se informa y se piden los datos correctos. Si no se proporcionan los datos, concluye el proceso. \\
	\end{tabularx}
\end{table}

\begin{table}[H]
	\begin{tabularx}{\textwidth}{X}
		\textbf{Otros datos}\\ \hline
	\end{tabularx}
	\begin{tabularx}{\textwidth}{lXlX}
		\textbf{Frecuencia esperada} &  & \textbf{Rendimiento} & \\
		\textbf{Importancia} & & \textbf{Urgencia} & \\
		\textbf{Estado} &  & \textbf{Estabilidad} & \\
	\end{tabularx}
	
	\bigskip
	
	\begin{tabularx}{\textwidth}{X}
		\textbf{Comentarios}\\ \hline
		El médico no entrega el medicamento al paciente, sino una receta con la que acudir a la farmacia y adquirir el medicamento. \\
	\end{tabularx}
\end{table}

\newpage




% IMPONER TRATAMIENTO %

\begin{table}[H]
	\begin{tabularx}{\textwidth}{l|Xlllr}
		\textbf{Caso de Uso}   & Imponer tratamiento & & & & \cu \\  
		\textbf{Actores}       & Médico, Paciente & & & & \\ 
		\textbf{Tipo}          & Secundario, esencial  & & & & \\
		%\textbf{Referencias}   & RF-9 & & CU-2 &\\
		\textbf{Referencias}   &  \multicolumn{5}{>{\hsize=\dimexpr\textwidth-\hsize\relax}X}{ RF-19 }\\
		\textbf{Precondición}  &  \multicolumn{5}{>{\hsize=\dimexpr\textwidth-0.85\hsize\relax}X}{ }\\ 
		\textbf{Postcondición} & \multicolumn{5}{>{\hsize=\dimexpr\textwidth-0.85\hsize\relax}X}{ Dejar reflejado en el sistema que el paciente está recibiendo dicho tratamiento. }\\
		\textbf{Autor}         & Simón López & \textbf{Fecha} & 20/04/18 & \textbf{Versión} & 1.0 \\ 
	\end{tabularx}
	
	\bigskip
	
	\begin{tabularx}{\textwidth}{X}
		\textbf{Propósito}\\ \hline
		Proporcionar al paciente un tratamiento médico para mejorar su estado de salud. \\
	\end{tabularx}
	
	\bigskip
	
	\begin{tabularx}{\textwidth}{X}
		\textbf{Resumen}\\ \hline
		El médico consulta los distintos tratamientos disponibles en la clínica, decidiendo cuál de todos es el más adecuando para tratar al paciente y mejorar su estado de salud. \\
	\end{tabularx}
	
	\bigskip
	
	\begin{tabularx}{\textwidth}{X}
		\textbf{Curso Normal (Básico)}\\ \hline
	\end{tabularx}
	\begin{tabularx}{\textwidth}{cXcX}
		\textbf{1} & El médico consulta los distintos tratamientos disponibles. & & \\
		\textbf{2} & El médico decide el tratamiento a imponer e introduce el tratamiento en la ficha del paciente.  & \textbf{3} & Se almacenan los nuevos datos. \\
		& & \textbf{4} & Se informa de que el proceso se ha realizado correctamente. \\
		\textbf{5} & Se informa al paciente del tratamiento asignado. & & \\
	\end{tabularx}
	
	\begin{tabularx}{\textwidth}{X}
		\textbf{Cursos Alternos}\\ \hline
	\end{tabularx}
	\begin{tabularx}{\textwidth}{cX}
		\textbf{2a} & Si el tratamiento introducido no está disponible en la clínica, se pide que se introduzca un nuevo tratamiento que sí lo esté; si no lo hace, se aborta el proceso. \\
	\end{tabularx}
\end{table}

\begin{table}[H]
	\begin{tabularx}{\textwidth}{X}
		\textbf{Otros datos}\\ \hline
	\end{tabularx}
	\begin{tabularx}{\textwidth}{lXlX}
		\textbf{Frecuencia esperada} &  & \textbf{Rendimiento} & \\
		\textbf{Importancia} & & \textbf{Urgencia} & \\
		\textbf{Estado} &  & \textbf{Estabilidad} & \\
	\end{tabularx}
	
	\bigskip
	
	\begin{tabularx}{\textwidth}{X}
		\textbf{Comentarios}\\ \hline
		[...] \\
	\end{tabularx}
\end{table}


\newpage




% PEDIR PRUEBA MÉDICA %

\begin{table}[H]
	\begin{tabularx}{\textwidth}{l|Xlllr}
		\textbf{Caso de Uso}   & Pedir prueba médica & & & & \cu \\  
		\textbf{Actores}       & Médico, Paciente & & & & \\ 
		\textbf{Tipo}          & Secundario, esencial  & & & & \\
		%\textbf{Referencias}   & RF-9 & & CU-2 &\\
		\textbf{Referencias}   &  \multicolumn{5}{>{\hsize=\dimexpr\textwidth-\hsize\relax}X}{ RNF-2 }\\
		\textbf{Precondición}  & \multicolumn{5}{>{\hsize=\dimexpr\textwidth-0.85\hsize\relax}X}{ La fecha elegida para la prueba está disponible. }\\ 
		\textbf{Postcondición} & \multicolumn{5}{>{\hsize=\dimexpr\textwidth-0.85\hsize\relax}X}{ Reflejar la fecha elegida como no disponible. }\\
		\textbf{Autor}         & Simón López & \textbf{Fecha} & 20/04/18 & \textbf{Versión} & 1.0 \\ 
	\end{tabularx}
	
	\bigskip
	
	\begin{tabularx}{\textwidth}{X}
		\textbf{Propósito}\\ \hline
		Pedir turno para realizar una prueba médica específica sobre el paciente. \\
	\end{tabularx}
	
	\bigskip
	
	\begin{tabularx}{\textwidth}{X}
		\textbf{Resumen}\\ \hline
		El médico accede al sistema y busca una posible fecha para realizar una prueba médica sobre el paciente, la cual realizará este mismo médico u otro especialista. \\
	\end{tabularx}
	
	\bigskip
	
	\begin{tabularx}{\textwidth}{X}
		\textbf{Curso Normal (Básico)}\\ \hline
	\end{tabularx}
	\begin{tabularx}{\textwidth}{cXcX}
		\textbf{1} & El médico consulta las posibles fechas para realizar la prueba. & & \\
		\textbf{2} & El médico acuerda con el paciente la mejor fecha posible y la introduce en el sistema.  & \textbf{3} & Se almacenan los nuevos datos. \\
		& & \textbf{4} & Se informa de que el proceso se ha realizado correctamente. \\
	\end{tabularx}
	
	\begin{tabularx}{\textwidth}{X}
		\textbf{Cursos Alternos}\\ \hline
	\end{tabularx}
	\begin{tabularx}{\textwidth}{cX}
		\textbf{2a} & Si la fecha acordada para realizar la prueba médica ya está ocupada, se pide que se introduzca una fecha distinta; si no se introduce, se aborta el procedimiento. \\
	\end{tabularx}
\end{table}

\begin{table}[H]
	\begin{tabularx}{\textwidth}{X}
		\textbf{Otros datos}\\ \hline
	\end{tabularx}
	\begin{tabularx}{\textwidth}{lXlX}
		\textbf{Frecuencia esperada} &  & \textbf{Rendimiento} & \\
		\textbf{Importancia} & & \textbf{Urgencia} & \\
		\textbf{Estado} &  & \textbf{Estabilidad} & \\
	\end{tabularx}
	
	\bigskip
	
	\begin{tabularx}{\textwidth}{X}
		\textbf{Comentarios}\\ \hline
		[...] \\
	\end{tabularx}
\end{table}


\newpage




% DESVIAR A ESPECIALISTA %

\begin{table}[H]
	\begin{tabularx}{\textwidth}{l|Xlllr}
		\textbf{Caso de Uso}   & Desviar a especialista. & & & & \cu \\  
		\textbf{Actores}       & Médico, Paciente & & & & \\ 
		\textbf{Tipo}          & Secundario, esencial  & & & & \\
		%\textbf{Referencias}   & RF-9 & & CU-2 &\\
		\textbf{Referencias}   &  \multicolumn{5}{>{\hsize=\dimexpr\textwidth-\hsize\relax}X}{ }\\
		\textbf{Precondición}  &  \multicolumn{5}{>{\hsize=\dimexpr\textwidth-0.85\hsize\relax}X}{ El especialista tiene disponibilidad para un nuevo paciente. }\\ 
		\textbf{Postcondición} & \multicolumn{5}{>{\hsize=\dimexpr\textwidth-0.85\hsize\relax}X}{ Dejar reflejado en el sistema que el paciente está con el especialista elegido. }\\
		\textbf{Autor}         & Simón López & \textbf{Fecha} & 20/04/18 & \textbf{Versión} & 1.0 \\ 
	\end{tabularx}
	
	\bigskip
	
	\begin{tabularx}{\textwidth}{X}
		\textbf{Propósito}\\ \hline
		Mejorar el estado de salud del paciente mediante un médico más especializado en su problema. \\
	\end{tabularx}
	
	\bigskip
	
	\begin{tabularx}{\textwidth}{X}
		\textbf{Resumen}\\ \hline
		Tras realizar un diagnóstico al paciente, el médico le asigna un especialista para profundizar más en la enfermedad que el paciente tenga, comunicándole al paciente quién será su nuevo médico. \\
	\end{tabularx}
	
	\bigskip
	
	\begin{tabularx}{\textwidth}{X}
		\textbf{Curso Normal (Básico)}\\ \hline
	\end{tabularx}
	\begin{tabularx}{\textwidth}{cXcX}
		\textbf{1} & El médico decide que el paciente debe de ser tratado por un especialista. & & \\
		\textbf{2} & El médico accede al sistema comprobando los especialistas disponibles para tratar la enfermedad del paciente e introduce el nuevo médico asignado para el paciente.  & \textbf{3} & Se almacenan los nuevos datos. \\
		& & \textbf{4} & Se informa de que el proceso se ha realizado correctamente. \\
		\textbf{5} & Se informa al paciente del nuevo médico asignado. & & \\
	\end{tabularx}
	
	\begin{tabularx}{\textwidth}{X}
		\textbf{Cursos Alternos}\\ \hline
	\end{tabularx}
	\begin{tabularx}{\textwidth}{cX}
		\textbf{2a} & Si el especialista asignado no tiene disponibilidad para un nuevo paciente, se pide que se le asigne otro especialista distinto; si no se le asigna, se aborta el proceso. \\
	\end{tabularx}
\end{table}

\begin{table}[H]
	\begin{tabularx}{\textwidth}{X}
		\textbf{Otros datos}\\ \hline
	\end{tabularx}
	\begin{tabularx}{\textwidth}{lXlX}
		\textbf{Frecuencia esperada} &  & \textbf{Rendimiento} & \\
		\textbf{Importancia} & & \textbf{Urgencia} & \\
		\textbf{Estado} &  & \textbf{Estabilidad} & \\
	\end{tabularx}
	
	\bigskip
	
	\begin{tabularx}{\textwidth}{X}
		\textbf{Comentarios}\\ \hline
		[...] \\
	\end{tabularx}
\end{table}
\newpage

% TERMINAR CONSULTA 

\begin{table}[H]
	\begin{tabularx}{\textwidth}{l|Xlllr}
		\textbf{Caso de Uso}   & Terminar consulta & & & & \cu \\  
		\textbf{Actores}       & Paciente, sanitario & & & \\ 
		\textbf{Tipo}          & Primario, esencial & & & \\
		\textbf{Referencias}   & \multicolumn{5}{>{\hsize=\dimexpr\textwidth-\hsize\relax}X}{Llamar siguiente paciente}\\
		\textbf{Precondición}  & \multicolumn{5}{>{\hsize=\dimexpr\textwidth-0.85\hsize\relax}X}{El sanitario ya ha tratado al paciente}\\ 
		\textbf{Postcondición} & \multicolumn{5}{>{\hsize=\dimexpr\textwidth-0.85\hsize\relax}X}{Paciente sale de la consulta}\\
		\textbf{Autor}         & Sofía Almeida & \textbf{Fecha} & 09/04/18 & \textbf{Versión} & 1.1 \\ 
	\end{tabularx}

        \bigskip

	\begin{tabularx}{\textwidth}{X}
		\textbf{Propósito}\\ \hline
                Si el paciente no tiene más molestias, abandona la consulta
	\end{tabularx}

	\bigskip

	\begin{tabularx}{\textwidth}{X}
		\textbf{Resumen}\\ \hline
		El sanitario pregunta al paciente si desea que traten más síntomas o problemas que éste pueda tener. Si responde de forma negativa se va de la consulta.
	\end{tabularx}

	\bigskip

	\begin{tabularx}{\textwidth}{X}
		\textbf{Curso Normal (Básico)}\\ \hline
	\end{tabularx}
	\begin{tabularx}{\textwidth}{cXcX}
		\textbf{1} & Sanitario: Pregunta al paciente si tiene alguna molestia más & & \\
		\textbf{2} & Paciente: Responde negativamente & \textbf{3} & Se cierra el historial clínico del paciente \\
		 & & \textbf{4} & Incluir (Llamar siguiente paciente) \\
	\end{tabularx}
        \begin{tabularx}{\textwidth}{X}
	  \textbf{Cursos Alternos}\\ \hline
	\end{tabularx}
	\begin{tabularx}{\textwidth}{cX}
	  \textbf{2a} & El paciente responde afirmativamente. El sanitario volvería a explorar al paciente, atendiendo a los nuevos síntomas descritos \\
	\end{tabularx}
\end{table}

\begin{table}[H]
	\begin{tabularx}{\textwidth}{X}
		\textbf{Otros datos}\\ \hline
	\end{tabularx}
	\begin{tabularx}{\textwidth}{lXlX}
		\textbf{Frecuencia esperada} & Cada media hora & \textbf{Rendimiento} & \\
		\textbf{Importancia} & Alta & \textbf{Urgencia} & Alta\\
	\end{tabularx}
	\begin{tabularx}{\textwidth}{X}
		\textbf{Comentarios}\\ \hline	
	\end{tabularx}
\end{table}
% FIN TERMINAR CONSULTA

% LLAMAR SIGUIENTE PACIENTE
\begin{table}[H]
	\begin{tabularx}{\textwidth}{l|Xlllr}
		\textbf{Caso de Uso}   & Llamar siguiente paciente & & & & \cu \\  
		\textbf{Actores}       & Sanitario, sistema de avisos & & & \\ 
		\textbf{Tipo}          & Primario, esencial & & & \\
		\textbf{Referencias}   & \multicolumn{5}{>{\hsize=\dimexpr\textwidth-\hsize\relax}X}{Terminar consulta}\\
		\textbf{Precondición}  & \multicolumn{5}{>{\hsize=\dimexpr\textwidth-0.85\hsize\relax}X}{La consulta ya ha terminado}\\ 
		\textbf{Postcondición} & \multicolumn{5}{>{\hsize=\dimexpr\textwidth-0.85\hsize\relax}X}{Habrá un nuevo aviso en el sistema de avisos}\\
		\textbf{Autor}         & Sofía Almeida & \textbf{Fecha} & 09/04/18 & \textbf{Versión} & 1.0 \\ 
	\end{tabularx}

	\bigskip

	\begin{tabularx}{\textwidth}{X}
		\textbf{Propósito}\\ \hline
                Notificar al siguiente paciente de que ya puede pasar a consulta
	\end{tabularx}

	\bigskip

	\begin{tabularx}{\textwidth}{X}
		\textbf{Resumen}\\ \hline
                El sanitario indica al sistema que quiere llamar a un nuevo paciente, el sistema buscará cuál es y se emitirá un aviso a través del sistema de avisos
        \end{tabularx}

	\bigskip

	\begin{tabularx}{\textwidth}{X}
		\textbf{Curso Normal (Básico)}\\ \hline
	\end{tabularx}
	\begin{tabularx}{\textwidth}{cXcX}
		\textbf{1} & Sanitario: Indica que quiere llamar a un nuevo paciente & & \\
	        & & \textbf{2} & Busca siguiente paciente \\
		& & \textbf{3} & Informa del siguiente paciente \\
		\textbf{4} & Sistema de avisos: Notifica al siguiente paciente & & \\
	\end{tabularx}
	
	\begin{tabularx}{\textwidth}{X}
	  \textbf{Cursos Alternos}\\ \hline
	\end{tabularx}
	\begin{tabularx}{\textwidth}{cX}
		\textbf{2a} & Si no hay más pacientes se espera hasta que sea la hora del siguiente paciente y continúa con el curso normal\\
	\end{tabularx}
\end{table}

\begin{table}[H]
	\begin{tabularx}{\textwidth}{X}
		\textbf{Otros datos}\\ \hline
	\end{tabularx}
	\begin{tabularx}{\textwidth}{lXlX}
		\textbf{Frecuencia esperada} & Cada media hora & \textbf{Rendimiento} & Alto en la mayor parte de los casos\\
		\textbf{Importancia} & Alta & \textbf{Urgencia} & Moderada\\
	\end{tabularx}
	
	\begin{tabularx}{\textwidth}{X}
		\textbf{Comentarios}\\ \hline
		El sistema de avisos puede ser muy diferente (pantalla, altavoz, ...) pero las acciones a realizar son las mismas
	\end{tabularx}
\end{table}

% FIN LLAMAR SIGUIENTE PACIENTE

% CONSULTAR HORARIO DE CONSULTA 

\begin{table}[H]
	\begin{tabularx}{\textwidth}{l|Xlllr}
		\textbf{Caso de Uso}   & Consultar horario de consulta & & & & \cu \\  
		\textbf{Actores}       & Usuario & & & \\ 
		\textbf{Tipo}          & Primario, esencial & & & \\
		\textbf{Referencias}   & \\
		\textbf{Precondición}  & \multicolumn{5}{>{\hsize=\dimexpr\textwidth-0.85\hsize\relax}X}{Sistema activo y operativo}\\ 
		\textbf{Postcondición} & \multicolumn{5}{>{\hsize=\dimexpr\textwidth-0.85\hsize\relax}X}{}\\
		\textbf{Autor}         & Sofía Almeida & \textbf{Fecha} & 11/04/18 & \textbf{Versión} & 1.0 \\ 
	\end{tabularx}

	\bigskip

	\begin{tabularx}{\textwidth}{X}
		\textbf{Propósito}\\ \hline
		Permitir que un paciente consulte el horario
	\end{tabularx}

	\bigskip

	\begin{tabularx}{\textwidth}{X}
		\textbf{Resumen}\\ \hline
		Un usuario entrará al sistema para ver en qué horario tiene una consulta médica, el sistema se lo mostrará
	\end{tabularx}

	\bigskip

	\begin{tabularx}{\textwidth}{X}
		\textbf{Curso Normal (Básico)}\\ \hline
	\end{tabularx}
	\begin{tabularx}{\textwidth}{cXcX}
		\textbf{1} & Usuario: Se identifica & & \\
		\textbf{2} & Usuario: Pregunta por su horario de consulta & & \\
		 & & \textbf{3} & Busca el horario de la consulta del usuario\\
		 & & \textbf{4} & Muestra al usuario el horario por el que preguntó\\
	\end{tabularx}
\end{table}

\begin{table}[H]
	\begin{tabularx}{\textwidth}{X}
		\textbf{Cursos Alternos}\\ \hline
	\end{tabularx}
	\begin{tabularx}{\textwidth}{cX}
	\end{tabularx}
	\begin{tabularx}{\textwidth}{X}
		\textbf{Otros datos}\\ \hline
	\end{tabularx}
	\begin{tabularx}{\textwidth}{lXlX}
		\textbf{Frecuencia esperada} & 2 veces antes de la consulta & \textbf{Rendimiento} & Alto\\
		\textbf{Importancia} & Moderada & \textbf{Urgencia} & Moderada\\
		\textbf{Estado} &  & \textbf{Estabilidad} & Alta \\
	\end{tabularx}
	\begin{tabularx}{\textwidth}{X}
		\textbf{Comentarios}\\ \hline
	\end{tabularx}
\end{table}

% FIN CONSULTAR HORARIO DE CONSULTA

% CREAR CONSULTA MÉDICA 

\begin{table}[H]
	\begin{tabularx}{\textwidth}{l|Xlllr}
		\textbf{Caso de Uso}   & Crear consulta médica & & & & \cu \\  
		\textbf{Actores}       & Administrativo & & & \\ 
		\textbf{Tipo}          & Primario, esencial & & & \\
		\textbf{Referencias}   & \multicolumn{5}{>{\hsize=\dimexpr\textwidth-\hsize\relax}X}{}\\
		\textbf{Precondición}  & \multicolumn{5}{>{\hsize=\dimexpr\textwidth-0.85\hsize\relax}X}{Plataforma activa y operativa}\\ 
		\textbf{Postcondición} & \multicolumn{5}{>{\hsize=\dimexpr\textwidth-0.85\hsize\relax}X}{Se ha añadido la consulta al sistema}\\
		\textbf{Autor}         & Sofía Almeida & \textbf{Fecha} & 11/04/18 & \textbf{Versión} & 1.0 \\ 
	\end{tabularx}

        \bigskip

	\begin{tabularx}{\textwidth}{X}
		\textbf{Propósito}\\ \hline
                Añadir una nueva consulta médica
	\end{tabularx}

	\bigskip

	\begin{tabularx}{\textwidth}{X}
		\textbf{Resumen}\\ \hline
		Un paciente pide una nueva consulta a un administrativo, que la añadirá al sistema
        \end{tabularx}

	\bigskip

	\begin{tabularx}{\textwidth}{X}
		\textbf{Curso Normal (Básico)}\\ \hline
	\end{tabularx}
	\begin{tabularx}{\textwidth}{cXcX}
		\textbf{1} & Paciente: pide una nueva consulta & & \\
		\textbf{2} & Administrativo: busca al paciente en el sistema & \textbf{3} & Muestra datos del paciente, entre ellos quién es su médico \\
		\textbf{4} & Administrativo: pregunta al paciente información sobre la consulta & & \\
          \textbf{5} & Paciente: comunica lo necesario & & \\
         \textbf{6} & Administrativo: introduce la información en el sistema &      \textbf{7} & Añade la consulta del paciente con toda la información necesaria
	\end{tabularx}
        \begin{tabularx}{\textwidth}{X}
	  \textbf{Cursos Alternos}\\ \hline
	\end{tabularx}
	\begin{tabularx}{\textwidth}{cX}
	\end{tabularx}
\end{table}

\begin{table}[H]
	\begin{tabularx}{\textwidth}{X}
		\textbf{Otros datos}\\ \hline
	\end{tabularx}
	\begin{tabularx}{\textwidth}{lXlX}
		\textbf{Frecuencia esperada} & Una vez al mes & \textbf{Rendimiento} & Medio-Alto\\
		\textbf{Importancia} & Alta & \textbf{Urgencia} & Alta\\
                \textbf{Estado} & & \textbf{Estabilidad} & Alta\\
	\end{tabularx}
	\begin{tabularx}{\textwidth}{X}
		\textbf{Comentarios}\\ \hline	
	\end{tabularx}
\end{table}
% FIN CREAR CONSULTA MÉDICA

% MODIFICAR CONSULTA MÉDICA
\begin{table}[H]
	\begin{tabularx}{\textwidth}{l|Xlllr}
		\textbf{Caso de Uso}   & Modificar consulta médica & & & & \cu \\  
		\textbf{Actores}       & Administrativo & & & \\ 
		\textbf{Tipo}          & Primario, esencial & & & \\
		\textbf{Referencias}   & \\
		\textbf{Precondición}  & \multicolumn{5}{>{\hsize=\dimexpr\textwidth-0.85\hsize\relax}X}{Debe haber una consulta ya creada}\\ 
		\textbf{Postcondición} & \multicolumn{5}{>{\hsize=\dimexpr\textwidth-0.85\hsize\relax}X}{Consulta actualizada}\\
		\textbf{Autor}         & Sofía Almeida & \textbf{Fecha} & 11/04/18 & \textbf{Versión} & 1.0 \\ 
	\end{tabularx}

	\bigskip

	\begin{tabularx}{\textwidth}{X}
		\textbf{Propósito}\\ \hline
                Realizar un cambio en una consulta ya establecida
	\end{tabularx}

	\bigskip

	\begin{tabularx}{\textwidth}{X}
		\textbf{Resumen}\\ \hline
                Un usuario indica a un administrativo que quiere realizar un cambio en su consulta, éste lo añade al sistema, que queda actualizado
        \end{tabularx}

	\bigskip

	\begin{tabularx}{\textwidth}{X}
		\textbf{Curso Normal (Básico)}\\ \hline
	\end{tabularx}
	\begin{tabularx}{\textwidth}{cXcX}
		\textbf{1} & Usuario: informa a un administrativo de que quiere realizar un cambio en una consulta médica & & \\
                \textbf{2} & Administrativo: solicita la modificación en la consulta & \textbf{3} & Localiza la consulta \\
		& & \textbf{4} & Realiza el cambio \\
		\textbf{5} & Administrativo: notifica al paciente que se ha realizado el cambio & & \\
	\end{tabularx}
	
	\begin{tabularx}{\textwidth}{X}
	  \textbf{Cursos Alternos}\\ \hline
	\end{tabularx}
	\begin{tabularx}{\textwidth}{cX}
	\end{tabularx}
\end{table}

\begin{table}[H]
	\begin{tabularx}{\textwidth}{X}
		\textbf{Otros datos}\\ \hline
	\end{tabularx}
	\begin{tabularx}{\textwidth}{lXlX}
		\textbf{Frecuencia esperada} & Baja & \textbf{Rendimiento} & Alto \\
		\textbf{Importancia} & Moderada & \textbf{Urgencia} & Moderada\\
                \textbf{Estado} & & \textbf{Estabilidad} & Alta \\
	\end{tabularx}
	
	\begin{tabularx}{\textwidth}{X}
		\textbf{Comentarios}\\ \hline
	\end{tabularx}
\end{table}

% FIN MODIFICAR CONSULTA MÉDICA

% ELIMINAR CONSULTA MÉDICA

\begin{table}[H]
	\begin{tabularx}{\textwidth}{l|Xlllr}
		\textbf{Caso de Uso}   & Eliminar consulta médica & & & & \cu \\  
		\textbf{Actores}       & Administrativo & & & \\ 
		\textbf{Tipo}          & Primario, esencial & & & \\
		\textbf{Referencias}   & \multicolumn{5}{>{\hsize=\dimexpr\textwidth-\hsize\relax}X}{}\\
		\textbf{Precondición}  & \multicolumn{5}{>{\hsize=\dimexpr\textwidth-0.85\hsize\relax}X}{Consulta existente}\\ 
		\textbf{Postcondición} & \multicolumn{5}{>{\hsize=\dimexpr\textwidth-0.85\hsize\relax}X}{Consulta eliminada del sistema}\\
		\textbf{Autor}         & Sofía Almeida & \textbf{Fecha} & 11/04/18 & \textbf{Versión} & 1.1 \\ 
	\end{tabularx}

	\bigskip

	\begin{tabularx}{\textwidth}{X}
		\textbf{Propósito}\\ \hline
		Eliminar una consulta médica
	\end{tabularx}

	\bigskip

	\begin{tabularx}{\textwidth}{X}
	  \textbf{Resumen}
          \\ \hline
          El administrativo decide que hay que eliminar una consulta médica, informa al sistema, que la borra del mismo
	\end{tabularx}

	\bigskip

	\begin{tabularx}{\textwidth}{X}
		\textbf{Curso Normal (Básico)}\\ \hline
	\end{tabularx}
	\begin{tabularx}{\textwidth}{cXcX}
		\textbf{1} & Administrativo: indica que quiere eliminar una consulta & & \\
		\textbf{2} & Administrativo: señala la consulta a eliminar & \textbf{3} & Busca la consulta y elimina toda su información \\
	\end{tabularx}
	
	\begin{tabularx}{\textwidth}{X}
		\textbf{Cursos Alternos}\\ \hline
	\end{tabularx}
	\begin{tabularx}{\textwidth}{cX}
	\end{tabularx}
\end{table}

\begin{table}[H]
	\begin{tabularx}{\textwidth}{X}
		\textbf{Otros datos}\\ \hline
	\end{tabularx}
	\begin{tabularx}{\textwidth}{lXlX}
		\textbf{Frecuencia esperada} & Una vez por consulta & \textbf{Rendimiento} & Alto\\
		\textbf{Importancia} & Alta & \textbf{Urgencia} & Alta\\
		\textbf{Estado} &  & \textbf{Estabilidad} & Alta\\
	\end{tabularx}
	
	\begin{tabularx}{\textwidth}{X}
		\textbf{Comentarios}\\ \hline
		El adiministrativo puede decirir que quiere eliminar una consulta por diferentes motivos: le informa un paciente, un médico, un paciente se da de baja, ...
	\end{tabularx}
\end{table}

% FIN ELIMINAR CONSULTA MÉDICA

% DEFINIR HORARIO DE CONSULTA MÉDICA

\begin{table}[H]
	\begin{tabularx}{\textwidth}{l|Xlllr}
		\textbf{Caso de Uso}   & Definir horario de consulta médica & & & & \cu \\  
		\textbf{Actores}       & Administrativo & & & \\ 
		\textbf{Tipo}          & Primario, esencial & & & \\
		\textbf{Referencias}   & \multicolumn{5}{>{\hsize=\dimexpr\textwidth-\hsize\relax}X}{}\\
		\textbf{Precondición}  & \multicolumn{5}{>{\hsize=\dimexpr\textwidth-0.85\hsize\relax}X}{Consulta médica en el sistema}\\ 
		\textbf{Postcondición} & \multicolumn{5}{>{\hsize=\dimexpr\textwidth-0.85\hsize\relax}X}{Consulta médica con horario definido}\\
		\textbf{Autor}         & Sofía Almeida & \textbf{Fecha} & 11/04/18 & \textbf{Versión} & 1.1 \\ 
	\end{tabularx}

	\bigskip

	\begin{tabularx}{\textwidth}{X}
		\textbf{Propósito}\\ \hline
		Añadir el horario a la consulta médica
	\end{tabularx}

	\bigskip

	\begin{tabularx}{\textwidth}{X}
	  \textbf{Resumen}\\ \hline
          El administrativo informa al paciente del horario de su médico para que decida en cuál quiere hacer dejar la consulta médica, posteriormente registra esta información en el sistema
	\end{tabularx}

	\bigskip

	\begin{tabularx}{\textwidth}{X}
		\textbf{Curso Normal (Básico)}\\ \hline
	\end{tabularx}
	\begin{tabularx}{\textwidth}{cXcX}
	  \textbf{1} & Administrativo: Busca el médico del paciente & & \\
          & & \textbf{2} & Devuelve la información solicitada \\
	  \textbf{3} & Administrativo: informa al usuario de los horarios de consulta disponible & \\
          \textbf{4} & Paciente: decide en qué horario quiere realizar la consulta e informa al administrativo \\
	  \textbf{5} & Administrativo: indica el horario decidido & & \\
		 & & \textbf{6} & Almacena el horario de la consulta\\
	\end{tabularx}
	
	\begin{tabularx}{\textwidth}{X}
		\textbf{Cursos Alternos}\\ \hline
	\end{tabularx}
	\begin{tabularx}{\textwidth}{cX}
	\end{tabularx}
\end{table}

\begin{table}[H]
	\begin{tabularx}{\textwidth}{X}
		\textbf{Otros datos}\\ \hline
	\end{tabularx}
	\begin{tabularx}{\textwidth}{lXlX}
		\textbf{Frecuencia esperada} & Una vez por consulta & \textbf{Rendimiento} & Alto \\
		\textbf{Importancia} & Alta & \textbf{Urgencia} & Alta \\
		\textbf{Estado} &  & \textbf{Estabilidad} & Alta\\
	\end{tabularx}
	
	\begin{tabularx}{\textwidth}{X}
		\textbf{Comentarios}\\ \hline
	\end{tabularx}
\end{table}

\newpage

% FIN DEFINIR HORARIO DE CONSULTA MÉDICA

% MODIFICAR HORARIO DE UNA CONSULTA
\begin{table}[H]
	\begin{tabularx}{\textwidth}{l|Xlllr}
		\textbf{Caso de Uso}   & Modificar horario de una consulta & & & & \cu \\  
		\textbf{Actores}       & Administrativo & & & \\ 
		\textbf{Tipo}          & Primario & & & \\
		\textbf{Referencias}   & \multicolumn{5}{>{\hsize=\dimexpr\textwidth-\hsize\relax}X}{}\\
		\textbf{Precondición}  & \multicolumn{5}{>{\hsize=\dimexpr\textwidth-0.85\hsize\relax}X}{Plataforma activa y operativa, consulta existente con horario ya definido}\\ 
		\textbf{Postcondición} & \multicolumn{5}{>{\hsize=\dimexpr\textwidth-0.85\hsize\relax}X}{Consulta actualizada en el sistema}\\
		\textbf{Autor}         & Sofía Almeida & \textbf{Fecha} & 20/04/18 & \textbf{Versión} & 1.0 \\ 
	\end{tabularx}

	\bigskip

	\begin{tabularx}{\textwidth}{X}
		\textbf{Propósito}\\ \hline
                Modificar el horario establecido para una consulta
	\end{tabularx}

	\bigskip

	\begin{tabularx}{\textwidth}{X}
		\textbf{Resumen}\\ \hline
                El administrativo busca una consulta y añade su nuevo horario.
        \end{tabularx}

	\bigskip

	\begin{tabularx}{\textwidth}{X}
		\textbf{Curso Normal (Básico)}\\ \hline
	\end{tabularx}
	\begin{tabularx}{\textwidth}{cXcX}
		\textbf{1} & Administrativo: busca la consulta & & \\
		 & & \textbf{2} & El sistema devuelve la consulta encontrada \\
		 \textbf{3} & Administrativo: indica que quiere modificar el horario de la consulta, junto al nuevo horario & & \\
		 & & \textbf{4} & El sistema cambia el horario guardado de esa consulta e indica que la operación se ha realizado con éxito\\
	\end{tabularx}
	
	\begin{tabularx}{\textwidth}{X}
		\textbf{Cursos Alternos}\\ \hline
	\end{tabularx}
	\begin{tabularx}{\textwidth}{cX}
		\textbf{4a} & Si el horario no está disponible, se vuelve al paso anterior\\
	\end{tabularx}
\end{table}

\begin{table}[H]
	\begin{tabularx}{\textwidth}{X}
		\textbf{Otros datos}\\ \hline
	\end{tabularx}
	\begin{tabularx}{\textwidth}{lXlX}
		\textbf{Frecuencia esperada} & 1 o 2 veces por consulta & \textbf{Rendimiento} & Alto\\
		\textbf{Importancia} & Moderada & \textbf{Urgencia} & Moderada\\
		\textbf{Estado} &  & \textbf{Estabilidad} & Alta\\
	\end{tabularx}
	
	\begin{tabularx}{\textwidth}{X}
		\textbf{Comentarios}\\ \hline
		El administrativo cambiará un horario porque se lo señala un sanitario, paciente, ...
	\end{tabularx}
\end{table}

% FIN MODIFICAR HORARIO CONSULTA

% MODIFICACIÓN PUNTUAL DEL HORARIO DE CONSULTA
\begin{table}[H]
	\begin{tabularx}{\textwidth}{l|Xlllr}
		\textbf{Caso de Uso}   & Modificación puntual horario de consulta & & & & \cu \\  
		\textbf{Actores}       & Administrativo & & & \\ 
		\textbf{Tipo}          & Primario & & & \\
		\textbf{Referencias}   & \multicolumn{5}{>{\hsize=\dimexpr\textwidth-\hsize\relax}X}{}\\
		\textbf{Precondición}  & \multicolumn{5}{>{\hsize=\dimexpr\textwidth-0.85\hsize\relax}X}{Plataforma activa y operativa, consulta existente con horario ya definido}\\ 
		\textbf{Postcondición} & \multicolumn{5}{>{\hsize=\dimexpr\textwidth-0.85\hsize\relax}X}{Consulta con indicador de que en alguna ocasión tendrá un horario diferente}\\
		\textbf{Autor}         & Sofía Almeida & \textbf{Fecha} & 20/04/18 & \textbf{Versión} & 1.0 \\ 
	\end{tabularx}

	\bigskip

	\begin{tabularx}{\textwidth}{X}
		\textbf{Propósito}\\ \hline
                Señalar que el horario de consulta se verá alterado una vez
        \end{tabularx}

	\bigskip

	\begin{tabularx}{\textwidth}{X}
		\textbf{Resumen}\\ \hline
                El administrativo busca la consulta y señala que se realizará un cambio no permanente en su horario
        \end{tabularx}

	\bigskip

	\begin{tabularx}{\textwidth}{X}
		\textbf{Curso Normal (Básico)}\\ \hline
	\end{tabularx}
	\begin{tabularx}{\textwidth}{cXcX}
		\textbf{1} & Administrativo: Busca la consulta & & \\
		 & & \textbf{2} & El sistema devuelve los datos de dicha consulta \\
		\textbf{3} & Administrativo: señala que quiere modificar el horario de una consulta en una fecha específica & \textbf{4} & El sistema comprueba el cambio, si es posible lo almacena y muestra que se ha realizado con éxito\\
	\end{tabularx}
	
	\begin{tabularx}{\textwidth}{X}
		\textbf{Cursos Alternos}\\ \hline
	\end{tabularx}
	\begin{tabularx}{\textwidth}{cX}
		\textbf{4a} & Si la modificación no es posible se vuelve al paso anterior, permitiendo que el administrativo realice la modificación a otro horario\\
	\end{tabularx}
\end{table}

\begin{table}[H]
	\begin{tabularx}{\textwidth}{X}
		\textbf{Otros datos}\\ \hline
	\end{tabularx}
	\begin{tabularx}{\textwidth}{lXlX}
		\textbf{Frecuencia esperada} & Baja & \textbf{Rendimiento} & Alto\\
		\textbf{Importancia} & Moderada & \textbf{Urgencia} & Moderada\\
		\textbf{Estado} &  & \textbf{Estabilidad} & Alta\\
	\end{tabularx}
	
	\begin{tabularx}{\textwidth}{X}
		\textbf{Comentarios}\\ \hline
	\end{tabularx}
\end{table}

% FIN MODIFICACIÓN PUNTAL HORARIO DE CONSULTA

% CONSULTAR DATOS PACIENTE
\begin{table}[H]
	\begin{tabularx}{\textwidth}{l|Xlllr}
		\textbf{Caso de Uso}   & Consultar datos de paciente & & & & \cu \\  
		\textbf{Actores}       & Usuario & & & \\ 
		\textbf{Tipo}          & Primario & & & \\
		\textbf{Referencias}   & \multicolumn{5}{>{\hsize=\dimexpr\textwidth-\hsize\relax}X}{}\\
		\textbf{Precondición}  & \multicolumn{5}{>{\hsize=\dimexpr\textwidth-0.85\hsize\relax}X}{Plataforma activa y operativa, paciente dado de alta en el sistema}\\ 
		\textbf{Postcondición} & \multicolumn{5}{>{\hsize=\dimexpr\textwidth-0.85\hsize\relax}X}{}\\
		\textbf{Autor}         & Sofía Almeida & \textbf{Fecha} & 20/04/18 & \textbf{Versión} & 1.0 \\ 
	\end{tabularx}

	\bigskip

	\begin{tabularx}{\textwidth}{X}
		\textbf{Propósito}\\ \hline
                Permitir a un usuario consultar datos de un paciente
	\end{tabularx}

	\bigskip

	\begin{tabularx}{\textwidth}{X}
		\textbf{Resumen}\\ \hline
                Un usuario accede al sistema e indica el identificador de un paciente, el sistema le devuelve sus datos
        \end{tabularx}

	\bigskip

	\begin{tabularx}{\textwidth}{X}
		\textbf{Curso Normal (Básico)}\\ \hline
	\end{tabularx}
	\begin{tabularx}{\textwidth}{cXcX}
		\textbf{1} & Usuario: señala la opción consultar datos de paciente & & \\
		\textbf{2} & Usuario: introduce el identificador de paciente & \textbf{3} & El sistema muestra los datos del paciente \\
	\end{tabularx}
	
	\begin{tabularx}{\textwidth}{X}
		\textbf{Cursos Alternos}\\ \hline
	\end{tabularx}
	\begin{tabularx}{\textwidth}{cX}
	\end{tabularx}
\end{table}

\begin{table}[H]
	\begin{tabularx}{\textwidth}{X}
		\textbf{Otros datos}\\ \hline
	\end{tabularx}
	\begin{tabularx}{\textwidth}{lXlX}
		\textbf{Frecuencia esperada} & Varias veces por paciente & \textbf{Rendimiento} & Alto\\
		\textbf{Importancia} & Alta & \textbf{Urgencia} & Moderada\\
		\textbf{Estado} &  & \textbf{Estabilidad} & Alta\\
	\end{tabularx}
	
	\begin{tabularx}{\textwidth}{X}
		\textbf{Comentarios}\\ \hline
	\end{tabularx}
\end{table}
% FIN CONSULTAR DATOS DE PACIENTE

% BAJA DE SANITARIO


\begin{table}[H]
	\begin{tabularx}{\textwidth}{l|Xlllr}
		\textbf{Caso de Uso}   & Baja de sanitario & & & & \cu \\  
		\textbf{Actores}       &  Sanitario, Compañía Médica & & & \\ 
		\textbf{Tipo}          & Secundario & & & \\
		\textbf{Referencias}   & \multicolumn{5}{>{\hsize=\dimexpr\textwidth-\hsize\relax}X}{RF-29, RN-1}\\
		\textbf{Precondición}  & \multicolumn{5}{>{\hsize=\dimexpr\textwidth-0.85\hsize\relax}X}{Plataforma activa y operativa}\\ 
		\textbf{Postcondición} & \multicolumn{5}{>{\hsize=\dimexpr\textwidth-0.85\hsize\relax}X}{Datos de sanitario borrados del sistema}\\
		\textbf{Autor}         &  Miguel Lentisco Ballesteros & \textbf{Fecha} & 18/04/18 & \textbf{Versión} & 1.0 \\ 
	\end{tabularx}

	\bigskip

	\begin{tabularx}{\textwidth}{X}
		\textbf{Propósito}\\ \hline
		Permite dar de baja a un sanitario, borrando sus datos del sistema.
	\end{tabularx}

	\bigskip

	\begin{tabularx}{\textwidth}{X}
		\textbf{Resumen}\\ \hline
		O el sanitario puede solicitar darse de baja o lo hace la propia compañía médica. La compañía médica se encarga de ejecutar la solicitud si procede y el sistema borra la información sobre el sanitario.
	\end{tabularx}

	\bigskip

	\begin{tabularx}{\textwidth}{X}
		\textbf{Curso Normal (Básico)}\\ \hline
	\end{tabularx}
	\begin{tabularx}{\textwidth}{cXcX}
		\textbf{1} & Sanitario: solicita a la compañía médica su baja & & \\
		\textbf{2} & Compañía médica: tramita la baja & & \\
		& & \textbf{3} & Elimina toda la información del sanitario \\
	\end{tabularx}
	
	\begin{tabularx}{\textwidth}{X}
		\textbf{Cursos Alternos}\\ \hline
	\end{tabularx}
	
		\begin{tabularx}{\textwidth}{cX}
		\textbf{1a} & Compañía médica: solicita la baja de un sanitario forzosamente.
	\end{tabularx}
\end{table}

\begin{table}[H]
	\begin{tabularx}{\textwidth}{X}
		\textbf{Otros datos}\\ \hline
	\end{tabularx}

	\begin{tabularx}{\textwidth}{lXlX}
		\textbf{Frecuencia esperada} & Baja & \textbf{Rendimiento} & Alto\\
		\textbf{Importancia} & Moderada & \textbf{Urgencia} & Baja\\
		\textbf{Estado} & En desarrollo & \textbf{Estabilidad} & Alta\\
	\end{tabularx}
	
	\begin{tabularx}{\textwidth}{X}
		\textbf{Comentarios}\\ \hline
	\end{tabularx}
\end{table}

\newpage

% FIN BAJA DE SANITARIO

% NUEVO RECURSO


\begin{table}[H]
	\begin{tabularx}{\textwidth}{l|Xlllr}
		\textbf{Caso de Uso}   & Nuevo recurso & & & & \cu \\  
		\textbf{Actores}       &  Administrativo & & & \\ 
		\textbf{Tipo}          & Primario & & & \\
		\textbf{Referencias}   & \multicolumn{5}{>{\hsize=\dimexpr\textwidth-\hsize\relax}X}{RF-28, RN-1}\\
		\textbf{Precondición}  & \multicolumn{5}{>{\hsize=\dimexpr\textwidth-0.85\hsize\relax}X}{Plataforma activa y operativa}\\ 
		\textbf{Postcondición} & \multicolumn{5}{>{\hsize=\dimexpr\textwidth-0.85\hsize\relax}X}{Recurso nuevo correctamente añadido}\\
		\textbf{Autor}         &  Miguel Lentisco Ballesteros & \textbf{Fecha} & 18/04/18 & \textbf{Versión} & 1.0 \\ 
	\end{tabularx}

	\bigskip

	\begin{tabularx}{\textwidth}{X}
		\textbf{Propósito}\\ \hline
		Permite al administrativo añadir un nuevo recurso (salas, espacios, maquinaria)
	\end{tabularx}

	\bigskip

	\begin{tabularx}{\textwidth}{X}
		\textbf{Resumen}\\ \hline
		El administrativo añade un nuevo recurso indicando el tipo, nombre, datos... y el sistema añade la información.
	\end{tabularx}

	\bigskip

	\begin{tabularx}{\textwidth}{X}
		\textbf{Curso Normal (Básico)}\\ \hline
	\end{tabularx}
	\begin{tabularx}{\textwidth}{cXcX}
		\textbf{1} & Administrativo: añade un nuevo recurso con la información pertinente & & \\
		& & \textbf{2} & Se registra el nuevo recurso y sus datos \\
	\end{tabularx}
	
	\begin{tabularx}{\textwidth}{X}
		\textbf{Cursos Alternos}\\ \hline
	\end{tabularx}
\end{table}

\begin{table}[H]
	\begin{tabularx}{\textwidth}{X}
		\textbf{Otros datos}\\ \hline
	\end{tabularx}

	\begin{tabularx}{\textwidth}{lXlX}
		\textbf{Frecuencia esperada} & Media & \textbf{Rendimiento} & Alto\\
		\textbf{Importancia} & Moderada & \textbf{Urgencia} & Baja\\
		\textbf{Estado} & En desarrollo & \textbf{Estabilidad} & Alta\\
	\end{tabularx}
	
	\begin{tabularx}{\textwidth}{X}
		\textbf{Comentarios}\\ \hline
	\end{tabularx}
\end{table}

\newpage

% FIN NUEVO RECURSO

% MODIFICAR RECURSO

\begin{table}[H]
	\begin{tabularx}{\textwidth}{l|Xlllr}
		\textbf{Caso de Uso}   & Modificar recurso & & & & \cu \\  
		\textbf{Actores}       &  Administrativo & & & \\ 
		\textbf{Tipo}          & Primario & & & \\
		\textbf{Referencias}   & \multicolumn{5}{>{\hsize=\dimexpr\textwidth-\hsize\relax}X}{RF-28, RN-1}\\
		\textbf{Precondición}  & \multicolumn{5}{>{\hsize=\dimexpr\textwidth-0.85\hsize\relax}X}{Plataforma activa y operativa}\\ 
		\textbf{Postcondición} & \multicolumn{5}{>{\hsize=\dimexpr\textwidth-0.85\hsize\relax}X}{Recurso modificado correctamente}\\
		\textbf{Autor}         &  Miguel Lentisco Ballesteros & \textbf{Fecha} & 18/04/18 & \textbf{Versión} & 1.0 \\ 
	\end{tabularx}

	\bigskip

	\begin{tabularx}{\textwidth}{X}
		\textbf{Propósito}\\ \hline
		Permite modificar un recurso ya existente en el sistema.
	\end{tabularx}

	\bigskip

	\begin{tabularx}{\textwidth}{X}
		\textbf{Resumen}\\ \hline
		El administrativo selecciona uno de los recursos ya existentes devueltos por el sistema y luego edita la información que desee, donde el sistema registra los cambios.  
	\end{tabularx}

	\bigskip

	\begin{tabularx}{\textwidth}{X}
		\textbf{Curso Normal (Básico)}\\ \hline
	\end{tabularx}
	\begin{tabularx}{\textwidth}{cXcX}
		\textbf{1} & Administrativo: solicita la lista de recursos disponibles & & \\
		& & \textbf{2} & Devuelve la lista de recursos \\
		\textbf{3} & Administrativo: selecciona el recurso de la lista deseado y modifica la información & & \\
		& & \textbf{4} & Se registran los cambios \\
	\end{tabularx}
	
	\begin{tabularx}{\textwidth}{X}
		\textbf{Cursos Alternos}\\ \hline
	\end{tabularx}
	
	\begin{tabularx}{\textwidth}{cX}
		\textbf{3a} & Si los cambios son incorrectos se descartan.
	\end{tabularx}
	
\end{table}

\begin{table}[H]
	\begin{tabularx}{\textwidth}{X}
		\textbf{Otros datos}\\ \hline
	\end{tabularx}

	\begin{tabularx}{\textwidth}{lXlX}
		\textbf{Frecuencia esperada} & Mediana & \textbf{Rendimiento} & Alto\\
		\textbf{Importancia} & Moderada & \textbf{Urgencia} & Mediana \\
		\textbf{Estado} & En desarrollo & \textbf{Estabilidad} & Alta\\
	\end{tabularx}
	
	\begin{tabularx}{\textwidth}{X}
		\textbf{Comentarios}\\ \hline
	\end{tabularx}
\end{table}

\newpage

% FIN MODIFICAR RECURSO

% BAJA DE RECURSO

\begin{table}[H]
	\begin{tabularx}{\textwidth}{l|Xlllr}
		\textbf{Caso de Uso}   & Baja de recurso & & & & \cu \\  
		\textbf{Actores}       &  Administrativo & & & \\ 
		\textbf{Tipo}          & Primario & & & \\
		\textbf{Referencias}   & \multicolumn{5}{>{\hsize=\dimexpr\textwidth-\hsize\relax}X}{RF-28, RN-1}\\
		\textbf{Precondición}  & \multicolumn{5}{>{\hsize=\dimexpr\textwidth-0.85\hsize\relax}X}{Plataforma activa y operativa}\\ 
		\textbf{Postcondición} & \multicolumn{5}{>{\hsize=\dimexpr\textwidth-0.85\hsize\relax}X}{Recurso eliminado correctamente}\\
		\textbf{Autor}         &  Miguel Lentisco Ballesteros & \textbf{Fecha} & 18/04/18 & \textbf{Versión} & 1.0 \\ 
	\end{tabularx}

	\bigskip

	\begin{tabularx}{\textwidth}{X}
		\textbf{Propósito}\\ \hline
		Permite eliminar un recurso ya existente en el sistema.
	\end{tabularx}

	\bigskip

	\begin{tabularx}{\textwidth}{X}
		\textbf{Resumen}\\ \hline
		El administrativo selecciona uno de los recursos ya existentes devueltos del sistema y el sistema elimina el recurso.  
	\end{tabularx}

	\bigskip

	\begin{tabularx}{\textwidth}{X}
		\textbf{Curso Normal (Básico)}\\ \hline
	\end{tabularx}
	\begin{tabularx}{\textwidth}{cXcX}
		\textbf{1} & Administrativo: solicita la lista de recursos disponibles & & \\
		& & \textbf{2} & Devuelve la lista de recursos \\
		\textbf{3} & Administrativo: selecciona el recurso de la lista deseado & & \\
		& & \textbf{4} & Se elimina el recurso seleccionado \\
	\end{tabularx}
	
	\begin{tabularx}{\textwidth}{X}
		\textbf{Cursos Alternos}\\ \hline
	\end{tabularx}
	
	
	
\end{table}

\begin{table}[H]
	\begin{tabularx}{\textwidth}{X}
		\textbf{Otros datos}\\ \hline
	\end{tabularx}

	\begin{tabularx}{\textwidth}{lXlX}
		\textbf{Frecuencia esperada} & Baja & \textbf{Rendimiento} & Alto\\
		\textbf{Importancia} & Moderada & \textbf{Urgencia} & Mediana \\
		\textbf{Estado} & En desarrollo & \textbf{Estabilidad} & Alta\\
	\end{tabularx}
	
	\begin{tabularx}{\textwidth}{X}
		\textbf{Comentarios}\\ \hline
	\end{tabularx}
\end{table}

\newpage

% FIN BAJA DE RECURSO

% ASIGNACIÓN PERMANENTE DE RECURSO

\begin{table}[H]
	\begin{tabularx}{\textwidth}{l|Xlllr}
		\textbf{Caso de Uso}   & Asignación permanente de recurso & & & & \cu \\  
		\textbf{Actores}       &  Administrativo & & & \\ 
		\textbf{Tipo}          & Primario & & & \\
		\textbf{Referencias}   & \multicolumn{5}{>{\hsize=\dimexpr\textwidth-\hsize\relax}X}{RF-28, RN-1}\\
		\textbf{Precondición}  & \multicolumn{5}{>{\hsize=\dimexpr\textwidth-0.85\hsize\relax}X}{Plataforma activa y operativa}\\ 
		\textbf{Postcondición} & \multicolumn{5}{>{\hsize=\dimexpr\textwidth-0.85\hsize\relax}X}{Recurso asignado permanentemente correctamente}\\
		\textbf{Autor}         &  Miguel Lentisco Ballesteros & \textbf{Fecha} & 18/04/18 & \textbf{Versión} & 1.0 \\ 
	\end{tabularx}

	\bigskip

	\begin{tabularx}{\textwidth}{X}
		\textbf{Propósito}\\ \hline
		Permite asignar permanentemente un recurso a otro.
	\end{tabularx}

	\bigskip

	\begin{tabularx}{\textwidth}{X}
		\textbf{Resumen}\\ \hline
		El administrativo selecciona uno de los recursos ya existentes sin asignar devueltos del sistema y después selecciona el recurso al que se le va asignar devuelto de los recursos compatibles disponibles, registrando la asignación.
	\end{tabularx}

	\bigskip

	\begin{tabularx}{\textwidth}{X}
		\textbf{Curso Normal (Básico)}\\ \hline
	\end{tabularx}
	\begin{tabularx}{\textwidth}{cXcX}
		\textbf{1} & Administrativo: solicita la lista de recursos disponibles & & \\
		& & \textbf{2} & Devuelve la lista de recursos \\
		\textbf{3} & Administrativo: selecciona el recurso de la lista deseado & & \\
		& & \textbf{4} & Devuelve la lista de recursos disponibles compatibles a los que asignar \\
		\textbf{5} & Administrativo: selecciona el recurso al que se va asignar & & \\
		& & \textbf{6} & Se hace la asignación permanente \\
	\end{tabularx}
	
	\begin{tabularx}{\textwidth}{X}
		\textbf{Cursos Alternos}\\ \hline
	\end{tabularx}
	
\end{table}

\begin{table}[H]
	\begin{tabularx}{\textwidth}{X}
		\textbf{Otros datos}\\ \hline
	\end{tabularx}

	\begin{tabularx}{\textwidth}{lXlX}
		\textbf{Frecuencia esperada} & Mediana & \textbf{Rendimiento} & Alto\\
		\textbf{Importancia} & Moderada & \textbf{Urgencia} & Mediana \\
		\textbf{Estado} & En desarrollo & \textbf{Estabilidad} & Alta\\
	\end{tabularx}
	
	\begin{tabularx}{\textwidth}{X}
		\textbf{Comentarios}\\ \hline
	\end{tabularx}
\end{table}

\newpage

% FIN ASIGNACIÓN PERMANENTE DE RECURSO

% ELIMINAR ASIGNACIÓN DE RECURSO

\begin{table}[H]
	\begin{tabularx}{\textwidth}{l|Xlllr}
		\textbf{Caso de Uso}   & Eliminar asignación de recurso & & & & \cu \\  
		\textbf{Actores}       &  Administrativo & & & \\ 
		\textbf{Tipo}          & Primario & & & \\
		\textbf{Referencias}   & \multicolumn{5}{>{\hsize=\dimexpr\textwidth-\hsize\relax}X}{RF-28, RN-1}\\
		\textbf{Precondición}  & \multicolumn{5}{>{\hsize=\dimexpr\textwidth-0.85\hsize\relax}X}{Plataforma activa y operativa}\\ 
		\textbf{Postcondición} & \multicolumn{5}{>{\hsize=\dimexpr\textwidth-0.85\hsize\relax}X}{Asignación de recurso eliminado correctamente}\\
		\textbf{Autor}         &  Miguel Lentisco Ballesteros & \textbf{Fecha} & 18/04/18 & \textbf{Versión} & 1.0 \\ 
	\end{tabularx}

	\bigskip

	\begin{tabularx}{\textwidth}{X}
		\textbf{Propósito}\\ \hline
		Permite eliminar la asignación de un recurso.
	\end{tabularx}

	\bigskip

	\begin{tabularx}{\textwidth}{X}
		\textbf{Resumen}\\ \hline
		El administrativo selecciona uno de los recursos asignados devueltos del sistema y éste elimina la asignación.
	\end{tabularx}

	\bigskip

	\begin{tabularx}{\textwidth}{X}
		\textbf{Curso Normal (Básico)}\\ \hline
	\end{tabularx}
	\begin{tabularx}{\textwidth}{cXcX}
		\textbf{1} & Administrativo: solicita la lista de recursos asignados & & \\
		& & \textbf{2} & Devuelve la lista de recursos asignados \\
		\textbf{3} & Administrativo: selecciona el recurso de la lista deseado & & \\
		& & \textbf{4} & Se hace la eliminación de la asignación \\
	\end{tabularx}
	
	\begin{tabularx}{\textwidth}{X}
		\textbf{Cursos Alternos}\\ \hline
	\end{tabularx}
	
\end{table}

\begin{table}[H]
	\begin{tabularx}{\textwidth}{X}
		\textbf{Otros datos}\\ \hline
	\end{tabularx}

	\begin{tabularx}{\textwidth}{lXlX}
		\textbf{Frecuencia esperada} & Mediana & \textbf{Rendimiento} & Alto\\
		\textbf{Importancia} & Moderada & \textbf{Urgencia} & Mediana \\
		\textbf{Estado} & En desarrollo & \textbf{Estabilidad} & Alta\\
	\end{tabularx}
	
	\begin{tabularx}{\textwidth}{X}
		\textbf{Comentarios}\\ \hline
	\end{tabularx}
\end{table}

\newpage

% FIN ELIMINAR ASIGNACIÓN DE RECURSO

% ASIGNACIÓN PUNTUAL DE RECURSO

\begin{table}[H]
	\begin{tabularx}{\textwidth}{l|Xlllr}
		\textbf{Caso de Uso}   & Asignación puntual de recurso & & & & \cu \\  
		\textbf{Actores}       &  Administrativo & & & \\ 
		\textbf{Tipo}          & Primario & & & \\
		\textbf{Referencias}   & \multicolumn{5}{>{\hsize=\dimexpr\textwidth-\hsize\relax}X}{RF-28, RN-1}\\
		\textbf{Precondición}  & \multicolumn{5}{>{\hsize=\dimexpr\textwidth-0.85\hsize\relax}X}{Plataforma activa y operativa}\\ 
		\textbf{Postcondición} & \multicolumn{5}{>{\hsize=\dimexpr\textwidth-0.85\hsize\relax}X}{Recurso asignado puntualmente correctamente}\\
		\textbf{Autor}         &  Miguel Lentisco Ballesteros & \textbf{Fecha} & 18/04/18 & \textbf{Versión} & 1.0 \\ 
	\end{tabularx}

	\bigskip

	\begin{tabularx}{\textwidth}{X}
		\textbf{Propósito}\\ \hline
		Permite asignar puntualmente un recurso a otro.
	\end{tabularx}

	\bigskip

	\begin{tabularx}{\textwidth}{X}
		\textbf{Resumen}\\ \hline
		El administrativo selecciona uno de los recursos ya existentes sin asignar devueltos del sistema y después selecciona el recurso al que se le va asignar devuelto de los recursos compatibles disponibles, registrando la asignación puntual indicando el tiempo de la asignación.
	\end{tabularx}

	\bigskip

	\begin{tabularx}{\textwidth}{X}
		\textbf{Curso Normal (Básico)}\\ \hline
	\end{tabularx}
	\begin{tabularx}{\textwidth}{cXcX}
		\textbf{1} & Administrativo: solicita la lista de recursos disponibles & & \\
		& & \textbf{2} & Devuelve la lista de recursos \\
		\textbf{3} & Administrativo: selecciona el recurso de la lista deseado y añade el tiempo de asignación & & \\
		& & \textbf{4} & Devuelve la lista de recursos disponibles compatibles a los que asignar \\
		\textbf{5} & Administrativo: selecciona el recurso al que se va asignar & & \\
		& & \textbf{6} & Se hace la asignación puntual durante el tiempo asignado \\
	\end{tabularx}
	
	\begin{tabularx}{\textwidth}{X}
		\textbf{Cursos Alternos}\\ \hline
	\end{tabularx}
	
\end{table}

\begin{table}[H]
	\begin{tabularx}{\textwidth}{X}
		\textbf{Otros datos}\\ \hline
	\end{tabularx}

	\begin{tabularx}{\textwidth}{lXlX}
		\textbf{Frecuencia esperada} & Mediana & \textbf{Rendimiento} & Alto\\
		\textbf{Importancia} & Moderada & \textbf{Urgencia} & Mediana \\
		\textbf{Estado} & En desarrollo & \textbf{Estabilidad} & Alta\\
	\end{tabularx}
	
	\begin{tabularx}{\textwidth}{X}
		\textbf{Comentarios}\\ \hline
	\end{tabularx}
\end{table}

\newpage

% FIN ASIGNACIÓN PUNTUAL DE RECURSO

% PEDIR INFORMACIÓN

\begin{table}[H]
	\begin{tabularx}{\textwidth}{l|Xlllr}
		\textbf{Caso de Uso}   & Pedir información & & & & \cu \\  
		\textbf{Actores}       &  Visitante y Administrativo & & & \\ 
		\textbf{Tipo}          & Primario & & & \\
		\textbf{Referencias}   & \multicolumn{5}{>{\hsize=\dimexpr\textwidth-\hsize\relax}X}{RF-19, RF-20, RF-21, RF-22, RF-23}\\
		\textbf{Precondición}  & \multicolumn{5}{>{\hsize=\dimexpr\textwidth-0.85\hsize\relax}X}{Plataforma activa y operativa}\\ 
		\textbf{Postcondición} & \multicolumn{5}{>{\hsize=\dimexpr\textwidth-0.85\hsize\relax}X}{Nada}\\
		\textbf{Autor}         &  Miguel Lentisco Ballesteros & \textbf{Fecha} & 18/04/18 & \textbf{Versión} & 1.0 \\ 
	\end{tabularx}

	\bigskip

	\begin{tabularx}{\textwidth}{X}
		\textbf{Propósito}\\ \hline
		Permite consultar la información deseada por el visitante.
	\end{tabularx}

	\bigskip

	\begin{tabularx}{\textwidth}{X}
		\textbf{Resumen}\\ \hline
		El visitante desea saber una información sobre un tema, que solicita al administrativo que es el que hace la consulta al sistema.
	\end{tabularx}

	\bigskip

	\begin{tabularx}{\textwidth}{X}
		\textbf{Curso Normal (Básico)}\\ \hline
	\end{tabularx}
	\begin{tabularx}{\textwidth}{cXcX}
		\textbf{1} & Visitante: hace la petición de información al administrativo & & \\
		\textbf{2} & Administrativo: toma la petición y hace la petición de información & & \\
		& & \textbf{3} & Devuelve la información de la consulta \\
		\textbf{4} & Administrativo: informa de la respuesta al visitante & & \\
	\end{tabularx}
	
	\begin{tabularx}{\textwidth}{X}
		\textbf{Cursos Alternos}\\ \hline
	\end{tabularx}
	
\end{table}

\begin{table}[H]
	\begin{tabularx}{\textwidth}{X}
		\textbf{Otros datos}\\ \hline
	\end{tabularx}

	\begin{tabularx}{\textwidth}{lXlX}
		\textbf{Frecuencia esperada} & Alta & \textbf{Rendimiento} & Alto\\
		\textbf{Importancia} & Moderada & \textbf{Urgencia} & Baja \\
		\textbf{Estado} & En desarrollo & \textbf{Estabilidad} & Alta\\
	\end{tabularx}
	
	\begin{tabularx}{\textwidth}{X}
		\textbf{Comentarios}\\ \hline
	\end{tabularx}
\end{table}

\newpage

% FIN PEDIR INFORMACIÓN	

% PEDIR CITA

\begin{table}[H]
	\begin{tabularx}{\textwidth}{l|Xlllr}
		\textbf{Caso de Uso}   & Pedir cita & & & & \cu \\  
		\textbf{Actores}       &  Paciente y Administrativo & & & \\ 
		\textbf{Tipo}          & Primario & & & \\
		\textbf{Referencias}   & \multicolumn{5}{>{\hsize=\dimexpr\textwidth-\hsize\relax}X}{RF-1, RF-7, RN-2}\\
		\textbf{Precondición}  & \multicolumn{5}{>{\hsize=\dimexpr\textwidth-0.85\hsize\relax}X}{Plataforma activa y operativa}\\ 
		\textbf{Postcondición} & \multicolumn{5}{>{\hsize=\dimexpr\textwidth-0.85\hsize\relax}X}{Cita añadida correctamente}\\
		\textbf{Autor}         &  Miguel Lentisco Ballesteros & \textbf{Fecha} & 18/04/18 & \textbf{Versión} & 1.0 \\ 
	\end{tabularx}

	\bigskip

	\begin{tabularx}{\textwidth}{X}
		\textbf{Propósito}\\ \hline
		Permite al paciente pedir una cita al administrativo.
	\end{tabularx}

	\bigskip

	\begin{tabularx}{\textwidth}{X}
		\textbf{Resumen}\\ \hline
		El paciente explica su problema al administrativo y éste elige el especialista que necesita, ofreciéndole al paciente una fecha que le pueda venir bien. Finalmente queda registrada la cita para el especialista en ese dia con ese paciente.
	\end{tabularx}

	\bigskip

	\begin{tabularx}{\textwidth}{X}
		\textbf{Curso Normal (Básico)}\\ \hline
	\end{tabularx}
	\begin{tabularx}{\textwidth}{cXcX}
		\textbf{1} & Paciente: explica su problema y que necesita una cita & & \\
		\textbf{2} & Administrativo: elige el área de  especialidad & & \\
		& & \textbf{3} & Devuelve la lista de médicos disponibles \\
		\textbf{4} & Administrativo: selecciona un médicos & & \\
		& & \textbf{5} & Muestra el horario disponible de ese médico \\
		\textbf{6} & Administrativo: escoge el día libre más próximo & & \\
		\textbf{7} & Paciente: confirma la fecha u elige otra & & \\
		\textbf{8} & Administrativo: confirma la cita médica & & \\
		& & \textbf{9} & Se registra correctamente la cita médica \\
	\end{tabularx}
	
	\begin{tabularx}{\textwidth}{X}
		\textbf{Cursos Alternos}\\ \hline
	\end{tabularx}
	
\end{table}

\begin{table}[H]
	\begin{tabularx}{\textwidth}{X}
		\textbf{Otros datos}\\ \hline
	\end{tabularx}

	\begin{tabularx}{\textwidth}{lXlX}
		\textbf{Frecuencia esperada} & Alta & \textbf{Rendimiento} & Alto\\
		\textbf{Importancia} & Moderada & \textbf{Urgencia} & Baja \\
		\textbf{Estado} & En desarrollo & \textbf{Estabilidad} & Alta\\
	\end{tabularx}
	
	\begin{tabularx}{\textwidth}{X}
		\textbf{Comentarios}\\ \hline
	\end{tabularx}
\end{table}

\newpage

% FIN PEDIR CITA

% ASISTIR CITA

\begin{table}[H]
	\begin{tabularx}{\textwidth}{l|Xlllr}
		\textbf{Caso de Uso}   & Asistir cita & & & & \cu \\  
		\textbf{Actores}       &  Paciente y Administrativo & & & \\ 
		\textbf{Tipo}          & Primario & & & \\
		\textbf{Referencias}   & \multicolumn{5}{>{\hsize=\dimexpr\textwidth-\hsize\relax}X}{RF-7}\\
		\textbf{Precondición}  & \multicolumn{5}{>{\hsize=\dimexpr\textwidth-0.85\hsize\relax}X}{Plataforma activa y operativa}\\ 
		\textbf{Postcondición} & \multicolumn{5}{>{\hsize=\dimexpr\textwidth-0.85\hsize\relax}X}{Cita asistida confirmada}\\
		\textbf{Autor}         &  Miguel Lentisco Ballesteros & \textbf{Fecha} & 18/04/18 & \textbf{Versión} & 1.0 \\ 
	\end{tabularx}

	\bigskip

	\begin{tabularx}{\textwidth}{X}
		\textbf{Propósito}\\ \hline
		Permite confirmar al administrativo la asistencia del paciente a una cita médica concertada.
	\end{tabularx}

	\bigskip

	\begin{tabularx}{\textwidth}{X}
		\textbf{Resumen}\\ \hline
		El paciente llega a la cita médica que había sido concertada y el administrativo lo confirma registrándolo.
	\end{tabularx}

	\bigskip

	\begin{tabularx}{\textwidth}{X}
		\textbf{Curso Normal (Básico)}\\ \hline
	\end{tabularx}
	\begin{tabularx}{\textwidth}{cXcX}
		\textbf{1} & Paciente: se presenta para la cita médica que tenía & & \\
		\textbf{2} & Administrativo: confirma la asistencia del paciente & & \\
		& & \textbf{3} & Se registra la asistencia del paciente a la cita médica.
	\end{tabularx}
	
	\begin{tabularx}{\textwidth}{X}
		\textbf{Cursos Alternos}\\ \hline
	\end{tabularx}
	
\end{table}

\begin{table}[H]
	\begin{tabularx}{\textwidth}{X}
		\textbf{Otros datos}\\ \hline
	\end{tabularx}

	\begin{tabularx}{\textwidth}{lXlX}
		\textbf{Frecuencia esperada} & Alta & \textbf{Rendimiento} & Alto\\
		\textbf{Importancia} & Moderada & \textbf{Urgencia} & Baja \\
		\textbf{Estado} & En desarrollo & \textbf{Estabilidad} & Alta\\
	\end{tabularx}
	
	\begin{tabularx}{\textwidth}{X}
		\textbf{Comentarios}\\ \hline
	\end{tabularx}
\end{table}

\newpage

% FIN ASISTIR CITA


% PEDIR CAMBIO DE CITA

\begin{table}[H]
	\begin{tabularx}{\textwidth}{l|Xlllr}
		\textbf{Caso de Uso}   & Pedir cambio de cita & & & & \cu \\  
		\textbf{Actores}       & Paciente, Administrativo & & & \\ 
		\textbf{Tipo}          & Secundario & & & \\
		\textbf{Referencias}   & \multicolumn{5}{>{\hsize=\dimexpr\textwidth-\hsize\relax}X}{RF-1, RF-5, RF-27, RN-2}\\
		\textbf{Precondición}  & \multicolumn{5}{>{\hsize=\dimexpr\textwidth-0.85\hsize\relax}X}{Plataforma activa y operativa, cita previa}\\ 
		\textbf{Postcondición} & \multicolumn{5}{>{\hsize=\dimexpr\textwidth-0.85\hsize\relax}X}{Cita modificada}\\
		\textbf{Autor}         & José María Martín Luque & \textbf{Fecha} & 17/04/18 & \textbf{Versión} & 1.0 \\ 
	\end{tabularx}

	\bigskip

	\begin{tabularx}{\textwidth}{X}
		\textbf{Propósito}\\ \hline
		Permitir al paciente cambiar las citas que tenga fijadas por si le ha surgido algún imprevisto en el momento de la cita.
	\end{tabularx}

	\bigskip

	\begin{tabularx}{\textwidth}{X}
		\textbf{Resumen}\\ \hline
		El paciente indica a un administrativo que quiere realizar un cambio de cita. Este último comprueba la disponibilidad y solicita al sistema una nueva cita, eliminando la anterior.
	\end{tabularx}

	\bigskip

	\begin{tabularx}{\textwidth}{X}
		\textbf{Curso Normal (Básico)}\\ \hline
	\end{tabularx}
	\begin{tabularx}{\textwidth}{cXcX}
		\textbf{1} & Paciente: solicita al administrativo el cambio de cita & & \\
		\textbf{2} & Administrativo: Comprueba disponibilidad y solicita al sistema el cambio de cita & \textbf{3} & Elimina la cita anterior y añade la nueva \\
	\end{tabularx}
	
	\begin{tabularx}{\textwidth}{X}
		\textbf{Cursos Alternos}\\ \hline
	\end{tabularx}
\end{table}

\begin{table}[H]
	\begin{tabularx}{\textwidth}{X}
		\textbf{Otros datos}\\ \hline
	\end{tabularx}

	\begin{tabularx}{\textwidth}{lXlX}
		\textbf{Frecuencia esperada} & Baja & \textbf{Rendimiento} & Alto\\
		\textbf{Importancia} & Moderada & \textbf{Urgencia} & Baja\\
		\textbf{Estado} &  & \textbf{Estabilidad} & Alta\\
	\end{tabularx}
	
	\begin{tabularx}{\textwidth}{X}
		\textbf{Comentarios}\\ \hline
	\end{tabularx}
\end{table}

\newpage

% FIN PEDIR CAMBIO DE CITA

% ANULAR CITA

\begin{table}[H]
	\begin{tabularx}{\textwidth}{l|Xlllr}
		\textbf{Caso de Uso}   & Anular cita & & & & \cu \\  
		\textbf{Actores}       & Paciente, Administrativo & & & \\ 
		\textbf{Tipo}          & Esencial & & & \\
		\textbf{Referencias}   & \multicolumn{5}{>{\hsize=\dimexpr\textwidth-\hsize\relax}X}{RF-5, RF-27, RN-2}\\
		\textbf{Precondición}  & \multicolumn{5}{>{\hsize=\dimexpr\textwidth-0.85\hsize\relax}X}{Plataforma activa y operativa, cita previa}\\ 
		\textbf{Postcondición} & \multicolumn{5}{>{\hsize=\dimexpr\textwidth-0.85\hsize\relax}X}{Cita eliminada}\\
		\textbf{Autor}         & José Mª Martín Luque & \textbf{Fecha} & 17/04/18 & \textbf{Versión} & 1.0 \\ 
	\end{tabularx}

	\bigskip

	\begin{tabularx}{\textwidth}{X}
		\textbf{Propósito}\\ \hline
		Permitir al paciente cancelar una cita por cualquier motivo por el que pueda ser necesario: disponibilidad, mejoría, cambio de centro, etc.
	\end{tabularx}

	\bigskip

	\begin{tabularx}{\textwidth}{X}
		\textbf{Resumen}\\ \hline
		El paciente indica a un administrativo que quiere cancelar su cita. El administrativo accede al sistema y elimina la cita.
	\end{tabularx}

	\bigskip

	\begin{tabularx}{\textwidth}{X}
		\textbf{Curso Normal (Básico)}\\ \hline
	\end{tabularx}
	\begin{tabularx}{\textwidth}{cXcX}
		\textbf{1} & Paciente: Solicita al administrativo que elimine la cita & & \\
		\textbf{2} & Administrativo: Accede al sistema y le indica que borre la cita & \textbf{3} & Elimina la cita \\
	\end{tabularx}
	
	\begin{tabularx}{\textwidth}{X}
		\textbf{Cursos Alternos}\\ \hline
	\end{tabularx}
\end{table}

\begin{table}[H]
	\begin{tabularx}{\textwidth}{X}
		\textbf{Otros datos}\\ \hline
	\end{tabularx}
	\begin{tabularx}{\textwidth}{lXlX}
		\textbf{Frecuencia esperada} & Baja & \textbf{Rendimiento} & Alto\\
		\textbf{Importancia} & Alta & \textbf{Urgencia} & Baja\\
		\textbf{Estado} &  & \textbf{Estabilidad} & Alta \\
	\end{tabularx}
	
	\begin{tabularx}{\textwidth}{X}
		\textbf{Comentarios}\\ \hline
	\end{tabularx}
\end{table}

\newpage

% FIN ANULAR CITA

% ENTREGAR VALORACIÓN

\begin{table}[H]
	\begin{tabularx}{\textwidth}{l|Xlllr}
		\textbf{Caso de Uso}   & Entregar valoración & & & & \cu \\  
		\textbf{Actores}       & Paciente, Administrativo & & & \\ 
		\textbf{Tipo}          & Secundario & & & \\
		\textbf{Referencias}   & \multicolumn{5}{>{\hsize=\dimexpr\textwidth-\hsize\relax}X}{-}\\
		\textbf{Precondición}  & \multicolumn{5}{>{\hsize=\dimexpr\textwidth-0.85\hsize\relax}X}{Plataforma activa y operativa}\\ 
		\textbf{Postcondición} & \multicolumn{5}{>{\hsize=\dimexpr\textwidth-0.85\hsize\relax}X}{}\\
		\textbf{Autor} & José María Martín Luque & \textbf{Fecha} & 17/04/18 & \textbf{Versión} & 1.0 \\ 
	\end{tabularx}

	\bigskip

	\begin{tabularx}{\textwidth}{X}
		\textbf{Propósito}\\ \hline
		Permitir al paciente valorar el servicio prestado.
	\end{tabularx}

	\bigskip

	\begin{tabularx}{\textwidth}{X}
		\textbf{Resumen}\\ \hline
		El paciente indica a un administrativo que quiere entregar una valoración. El administrativo la recoge y la introduce en el sistema.
	\end{tabularx}

	\bigskip

	\begin{tabularx}{\textwidth}{X}
		\textbf{Curso Normal (Básico)}\\ \hline
	\end{tabularx}
	\begin{tabularx}{\textwidth}{cXcX}
		\textbf{1} & Paciente: Entrega al administrativo la valoración & & \\
		\textbf{2} & Administrativo: Accede al sistema y la introduce la valoración & \textbf{3} & Procesa la valoración \\
	\end{tabularx}
	
	\begin{tabularx}{\textwidth}{X}
		\textbf{Cursos Alternos}\\ \hline
	\end{tabularx}
\end{table}

\begin{table}[H]
	\begin{tabularx}{\textwidth}{X}
		\textbf{Otros datos}\\ \hline
	\end{tabularx}
	\begin{tabularx}{\textwidth}{lXlX}
		\textbf{Frecuencia esperada} & Muy baja & \textbf{Rendimiento} & Alto\\
		\textbf{Importancia} & Media & \textbf{Urgencia} & Baja\\
		\textbf{Estado} &  & \textbf{Estabilidad} & Alta \\
	\end{tabularx}
	
	\begin{tabularx}{\textwidth}{X}
		\textbf{Comentarios}\\ \hline
	\end{tabularx}
\end{table}

\newpage

% FIN ENTREGAR VALORACIÓN

% PEDIR CITA REMOTO

\begin{table}[H]
	\begin{tabularx}{\textwidth}{l|Xlllr}
		\textbf{Caso de Uso}   & Pedir cita remoto & & & & \cu \\  
		\textbf{Actores}       & Paciente & & & \\ 
		\textbf{Tipo}          & Esencial & & & \\
		\textbf{Referencias}   & \multicolumn{5}{>{\hsize=\dimexpr\textwidth-\hsize\relax}X}{RF-1, RF-3, RF-4, RN-3}\\
		\textbf{Precondición}  & \multicolumn{5}{>{\hsize=\dimexpr\textwidth-0.85\hsize\relax}X}{Plataforma activa y operativa}\\ 
		\textbf{Postcondición} & \multicolumn{5}{>{\hsize=\dimexpr\textwidth-0.85\hsize\relax}X}{Cita guardada}\\
		\textbf{Autor}         & José Mª Martín Luque & \textbf{Fecha} & 17/04/18 & \textbf{Versión} & 1.0 \\ 
	\end{tabularx}

	\bigskip

	\begin{tabularx}{\textwidth}{X}
		\textbf{Propósito}\\ \hline
		Permitir al paciente pedir cita a través del servicio web escogiendo hora (si hay disponibilidad), médico y centro.
	\end{tabularx}

	\bigskip

	\begin{tabularx}{\textwidth}{X}
		\textbf{Resumen}\\ \hline
		El paciente accede a la plataforma web con sus credenciales y entra en el apartado de pedir cita. Introduce la fecha, hora, centro, especialidad y médico deseados y confirma la cita.
	\end{tabularx}

	\bigskip

	\begin{tabularx}{\textwidth}{X}
		\textbf{Curso Normal (Básico)}\\ \hline
	\end{tabularx}
	\begin{tabularx}{\textwidth}{cXcX}
		\textbf{1} & Paciente: Accede al servicio web & & \\
		\textbf{2} & Paciente: Accede al apartado de pedir cita & & \\
		\textbf{3} & Paciente: Indica especialidad, fecha y hora & \textbf{4} & Comprueba la disponibilidad del momento especificado\\
		\textbf{5} & Paciente: Confirma la cita & \textbf{6} & Almacena la cita \\
	\end{tabularx}
	
	\begin{tabularx}{\textwidth}{X}
		\textbf{Cursos Alternos}\\ \hline
			\textbf{5a} Paciente: Si no hay disponibilidad, volvemos al paso anterior & \\
	\end{tabularx}
\end{table}

\begin{table}[H]
	\begin{tabularx}{\textwidth}{X}
		\textbf{Otros datos}\\ \hline
	\end{tabularx}
	\begin{tabularx}{\textwidth}{lXlX}
		\textbf{Frecuencia esperada} & Alta, varias veces al día & \textbf{Rendimiento} & Alto\\
		\textbf{Importancia} & Alta & \textbf{Urgencia} & Alta\\
		\textbf{Estado} &  & \textbf{Estabilidad} & Alta \\
	\end{tabularx}
	
	\begin{tabularx}{\textwidth}{X}
		\textbf{Comentarios}\\ \hline
	\end{tabularx}
\end{table}

\newpage

% FIN PEDIR CITA REMOTO

% PEDIR CAMBIO DE CITA REMOTO

\begin{table}[H]
	\begin{tabularx}{\textwidth}{l|Xlllr}
		\textbf{Caso de Uso}   & Pedir cambio de cita remoto & & & & \cu \\  
		\textbf{Actores}       & Paciente & & & \\ 
		\textbf{Tipo}          & Secundario & & & \\
		\textbf{Referencias}   & \multicolumn{5}{>{\hsize=\dimexpr\textwidth-\hsize\relax}X}{RF-1, RF-5, RF-27, RN-2}\\
		\textbf{Precondición}  & \multicolumn{5}{>{\hsize=\dimexpr\textwidth-0.85\hsize\relax}X}{Plataforma activa y operativa, cita previa}\\ 
		\textbf{Postcondición} & \multicolumn{5}{>{\hsize=\dimexpr\textwidth-0.85\hsize\relax}X}{Cita modificada}\\
		\textbf{Autor}         & José Mª Martín Luque & \textbf{Fecha} & 17/04/18 & \textbf{Versión} & 1.0 \\ 
	\end{tabularx}

	\bigskip

	\begin{tabularx}{\textwidth}{X}
		\textbf{Propósito}\\ \hline
		Permitir al paciente cambiar las citas que tenga fijadas por si le ha surgido algún imprevisto en el momento de la cita.
	\end{tabularx}

	\bigskip

	\begin{tabularx}{\textwidth}{X}
		\textbf{Resumen}\\ \hline
		El paciente accede al apartado de citas del servicio web e infica al sistema que quiere realizar un cambio de cita. El sistema comprueba la disponibilidad y elimina la cita anterior.
	\end{tabularx}

	\bigskip

	\begin{tabularx}{\textwidth}{X}
		\textbf{Curso Normal (Básico)}\\ \hline
	\end{tabularx}
	\begin{tabularx}{\textwidth}{cXcX}
		\textbf{1} & Paciente: Accede al servicio web & & \\
		\textbf{2} & Paciente: Accede al apartado de citas & & \\
		\textbf{3} & Paciente: Indica la cita que quiere cambiar & & \\
		\textbf{4} & Paciente: Indica nueva fecha y hora & \textbf{5} & Comprueba la disponibilidad del momento especificado\\
		\textbf{6} & Paciente: Confirma la nueva cita & \textbf{7} & Almacena la nueva cita \\
	\end{tabularx}
	
	\begin{tabularx}{\textwidth}{X}
		\textbf{Cursos Alternos}\\ \hline
		\textbf{6a} Paciente: Si no hay disponibilidad, volvemos al paso anterior & \\
	\end{tabularx}
\end{table}

\begin{table}[H]
	\begin{tabularx}{\textwidth}{X}
		\textbf{Otros datos}\\ \hline
	\end{tabularx}

	\begin{tabularx}{\textwidth}{lXlX}
		\textbf{Frecuencia esperada} & Baja & \textbf{Rendimiento} & Alto\\
		\textbf{Importancia} & Moderada & \textbf{Urgencia} & Baja\\
		\textbf{Estado} &  & \textbf{Estabilidad} & Alta\\
	\end{tabularx}
	
	\begin{tabularx}{\textwidth}{X}
		\textbf{Comentarios}\\ \hline
	\end{tabularx}
\end{table}

\newpage

% FIN PEDIR CAMBIO DE CITA REMOTO

% ANULAR CITA REMOTO

\begin{table}[H]
	\begin{tabularx}{\textwidth}{l|Xlllr}
		\textbf{Caso de Uso}   & Anular cita remoto & & & \cu \\  
		\textbf{Actores}       & Paciente& & & \\ 
		\textbf{Tipo}          & Esencial & & & \\
		\textbf{Referencias}   & \multicolumn{5}{>{\hsize=\dimexpr\textwidth-\hsize\relax}X}{RF-5, RF-27, RN-2}\\
		\textbf{Precondición}  & \multicolumn{5}{>{\hsize=\dimexpr\textwidth-0.85\hsize\relax}X}{Plataforma activa y operativa, cita previa}\\ 
		\textbf{Postcondición} & \multicolumn{5}{>{\hsize=\dimexpr\textwidth-0.85\hsize\relax}X}{Cita eliminada}\\
		\textbf{Autor}         & José Mª Martín Luque & \textbf{Fecha} & 17/04/18 & \textbf{Versión} & 1.0 \\ 
	\end{tabularx}

	\bigskip

	\begin{tabularx}{\textwidth}{X}
		\textbf{Propósito}\\ \hline
		Permitir al paciente cancelar una cita por cualquier motivo por el que pueda ser necesario: disponibilidad, mejoría, cambio de centro, etc.
	\end{tabularx}

	\bigskip

	\begin{tabularx}{\textwidth}{X}
		\textbf{Resumen}\\ \hline
		El paciente accede al apartado de citas del servicio web e infica al sistema que quiere cancelar una cita. El sistema elimina la cita.
	\end{tabularx}

	\bigskip

	\begin{tabularx}{\textwidth}{X}
		\textbf{Curso Normal (Básico)}\\ \hline
	\end{tabularx}
	\begin{tabularx}{\textwidth}{cXcX}
		\textbf{1} & Paciente: Accede al servicio web & & \\
		\textbf{2} & Paciente: Accede al apartado de citas & & \\
		\textbf{3} & Paciente: Indica la cita que quiere eliminar & \textbf{4} & Elimina la cita \\
	\end{tabularx}
	
	\begin{tabularx}{\textwidth}{X}
		\textbf{Cursos Alternos}\\ \hline
	\end{tabularx}
\end{table}

\begin{table}[H]
	\begin{tabularx}{\textwidth}{X}
		\textbf{Otros datos}\\ \hline
	\end{tabularx}
	\begin{tabularx}{\textwidth}{lXlX}
		\textbf{Frecuencia esperada} & Baja & \textbf{Rendimiento} & Alto\\
		\textbf{Importancia} & Alta & \textbf{Urgencia} & Baja\\
		\textbf{Estado} &  & \textbf{Estabilidad} & Alta \\
	\end{tabularx}
	
	\begin{tabularx}{\textwidth}{X}
		\textbf{Comentarios}\\ \hline
	\end{tabularx}
\end{table}

\newpage

% FIN ANULAR CITA REMOTO

% VALORACIÓN REMOTA

\begin{table}[H]
	\begin{tabularx}{\textwidth}{l|Xlllr}
		\textbf{Caso de Uso}   & Valoración remota & & & & \cu \\  
		\textbf{Actores}       & Paciente & & & \\ 
		\textbf{Tipo}          & Secundario & & & \\
		\textbf{Referencias}   & \multicolumn{5}{>{\hsize=\dimexpr\textwidth-\hsize\relax}X}{-}\\
		\textbf{Precondición}  & \multicolumn{5}{>{\hsize=\dimexpr\textwidth-0.85\hsize\relax}X}{Plataforma activa y operativa}\\ 
		\textbf{Postcondición} & \multicolumn{5}{>{\hsize=\dimexpr\textwidth-0.85\hsize\relax}X}{}\\
		\textbf{Autor} & José Mª Martín Luque & \textbf{Fecha} & 17/04/18 & \textbf{Versión} & 1.0 \\ 
	\end{tabularx}

	\bigskip

	\begin{tabularx}{\textwidth}{X}
		\textbf{Propósito}\\ \hline
		Permitir al paciente valorar el servicio prestado.
	\end{tabularx}

	\bigskip

	\begin{tabularx}{\textwidth}{X}
		\textbf{Resumen}\\ \hline
		El paciente accede al apartado de valoraciones del sistema e indica a lo que quiere valorar. El sistema almacena y procesa la valoración.
	\end{tabularx}

	\bigskip

	\begin{tabularx}{\textwidth}{X}
		\textbf{Curso Normal (Básico)}\\ \hline
	\end{tabularx}
	\begin{tabularx}{\textwidth}{cXcX}
		\textbf{1} & Paciente: Accede al servicio web & & \\
		\textbf{2} & Paciente: Accede al apartado de valoraciones & & \\
		\textbf{3} & Paciente: Selecciona el sanitario, la consulta o la unidad de diagnóstico que quiere valorar. & \textbf{4} & Almacena y procesa la valoración \\
	\end{tabularx}
	
	\begin{tabularx}{\textwidth}{X}
		\textbf{Cursos Alternos}\\ \hline
	\end{tabularx}
\end{table}

\begin{table}[H]
	\begin{tabularx}{\textwidth}{X}
		\textbf{Otros datos}\\ \hline
	\end{tabularx}
	\begin{tabularx}{\textwidth}{lXlX}
		\textbf{Frecuencia esperada} & Muy baja & \textbf{Rendimiento} & Alto\\
		\textbf{Importancia} & Media & \textbf{Urgencia} & Baja\\
		\textbf{Estado} &  & \textbf{Estabilidad} & Alta \\
	\end{tabularx}
	
	\begin{tabularx}{\textwidth}{X}
		\textbf{Comentarios}\\ \hline
	\end{tabularx}
\end{table}

\newpage

% FIN VALORACIÓN REMOTA

% VALORACIÓN REMOTA DE SANITARIO

\begin{table}[H]
	\begin{tabularx}{\textwidth}{l|Xlllr}
		\textbf{Caso de Uso}   & Valoración remota de sanitario& & & & \cu \\  
		\textbf{Actores}       & Paciente & & & \\ 
		\textbf{Tipo}          & Secundario & & & \\
		\textbf{Referencias}   & \multicolumn{5}{>{\hsize=\dimexpr\textwidth-\hsize\relax}X}{Valoración remota}\\
		\textbf{Precondición}  & \multicolumn{5}{>{\hsize=\dimexpr\textwidth-0.85\hsize\relax}X}{Plataforma activa y operativa}\\ 
		\textbf{Postcondición} & \multicolumn{5}{>{\hsize=\dimexpr\textwidth-0.85\hsize\relax}X}{}\\
		\textbf{Autor} & José Mª Martín Luque & \textbf{Fecha} & 17/04/18 & \textbf{Versión} & 1.0 \\ 
	\end{tabularx}

	\bigskip

	\begin{tabularx}{\textwidth}{X}
		\textbf{Propósito}\\ \hline
		Permitir al paciente valorar el servicio prestado por un sanitario.
	\end{tabularx}

	\bigskip

	\begin{tabularx}{\textwidth}{X}
		\textbf{Resumen}\\ \hline
		El paciente accede al apartado de valoraciones del sistema e indica a el sanitario que quiere valorar. El sistema almacena y procesa la valoración.
	\end{tabularx}

	\bigskip

	\begin{tabularx}{\textwidth}{X}
		\textbf{Curso Normal (Básico)}\\ \hline
	\end{tabularx}
	\begin{tabularx}{\textwidth}{cXcX}
		\textbf{1} & Paciente: Accede al servicio web & & \\
		\textbf{2} & Paciente: Accede al apartado de valoraciones & & \\
		\textbf{3} & Paciente: Selecciona el sanitario que quiere valorar. & \textbf{4} & Almacena y procesa la valoración \\
	\end{tabularx}
	
	\begin{tabularx}{\textwidth}{X}
		\textbf{Cursos Alternos}\\ \hline
	\end{tabularx}
\end{table}

\begin{table}[H]
	\begin{tabularx}{\textwidth}{X}
		\textbf{Otros datos}\\ \hline
	\end{tabularx}
	\begin{tabularx}{\textwidth}{lXlX}
		\textbf{Frecuencia esperada} & Muy baja & \textbf{Rendimiento} & Alto\\
		\textbf{Importancia} & Media & \textbf{Urgencia} & Baja\\
		\textbf{Estado} &  & \textbf{Estabilidad} & Alta \\
	\end{tabularx}
	
	\begin{tabularx}{\textwidth}{X}
		\textbf{Comentarios}\\ \hline
	\end{tabularx}
\end{table}

\newpage

% FIN VALORACIÓN REMOTA DE SANITARIO

% VALORACIÓN REMOTA DE CONSULTA

\begin{table}[H]
	\begin{tabularx}{\textwidth}{l|Xlllr}
		\textbf{Caso de Uso}   & Valoración remota de consulta & & & & \cu \\  
		\textbf{Actores}       & Paciente & & & \\ 
		\textbf{Tipo}          & Secundario & & & \\
		\textbf{Referencias}   & \multicolumn{5}{>{\hsize=\dimexpr\textwidth-\hsize\relax}X}{Valoración remota}\\
		\textbf{Precondición}  & \multicolumn{5}{>{\hsize=\dimexpr\textwidth-0.85\hsize\relax}X}{Plataforma activa y operativa}\\ 
		\textbf{Postcondición} & \multicolumn{5}{>{\hsize=\dimexpr\textwidth-0.85\hsize\relax}X}{}\\
		\textbf{Autor} & José Mª Martín Luque & \textbf{Fecha} & 17/04/18 & \textbf{Versión} & 1.0 \\ 
	\end{tabularx}

	\bigskip

	\begin{tabularx}{\textwidth}{X}
		\textbf{Propósito}\\ \hline
		Permitir al paciente valorar el servicio prestado por el personal en una consulta.
	\end{tabularx}

	\bigskip

	\begin{tabularx}{\textwidth}{X}
		\textbf{Resumen}\\ \hline
		El paciente accede al apartado de valoraciones del sistema e indica la consulta que quiere valorar. El sistema almacena y procesa la valoración.
	\end{tabularx}

	\bigskip

	\begin{tabularx}{\textwidth}{X}
		\textbf{Curso Normal (Básico)}\\ \hline
	\end{tabularx}
	\begin{tabularx}{\textwidth}{cXcX}
		\textbf{1} & Paciente: Accede al servicio web & & \\
		\textbf{2} & Paciente: Accede al apartado de valoraciones & & \\
		\textbf{3} & Paciente: Selecciona la consulta que quiere valorar. & \textbf{4} & Almacena y procesa la valoración \\
	\end{tabularx}
	
	\begin{tabularx}{\textwidth}{X}
		\textbf{Cursos Alternos}\\ \hline
	\end{tabularx}
\end{table}

\begin{table}[H]
	\begin{tabularx}{\textwidth}{X}
		\textbf{Otros datos}\\ \hline
	\end{tabularx}
	\begin{tabularx}{\textwidth}{lXlX}
		\textbf{Frecuencia esperada} & Muy baja & \textbf{Rendimiento} & Alto\\
		\textbf{Importancia} & Media & \textbf{Urgencia} & Baja\\
		\textbf{Estado} &  & \textbf{Estabilidad} & Alta \\
	\end{tabularx}
	
	\begin{tabularx}{\textwidth}{X}
		\textbf{Comentarios}\\ \hline
	\end{tabularx}
\end{table}

\newpage

% FIN VALORACIÓN REMOTA DE CONSULTA

% VALORACIÓN REMOTA DE UNIDAD DE DIAGNOSTICO

\begin{table}[H]
	\begin{tabularx}{\textwidth}{l|Xlllr}
		\textbf{Caso de Uso}   & Valoración remota de unidad de diagnóstico& & & & \cu \\  
		\textbf{Actores}       & Paciente & & & \\ 
		\textbf{Tipo}          & Secundario & & & \\
		\textbf{Referencias}   & \multicolumn{5}{>{\hsize=\dimexpr\textwidth-\hsize\relax}X}{Valoración remota}\\
		\textbf{Precondición}  & \multicolumn{5}{>{\hsize=\dimexpr\textwidth-0.85\hsize\relax}X}{Plataforma activa y operativa}\\ 
		\textbf{Postcondición} & \multicolumn{5}{>{\hsize=\dimexpr\textwidth-0.85\hsize\relax}X}{}\\
		\textbf{Autor} & José Mª Martín Luque & \textbf{Fecha} & 17/04/18 & \textbf{Versión} & 1.0 \\ 
	\end{tabularx}

	\bigskip

	\begin{tabularx}{\textwidth}{X}
		\textbf{Propósito}\\ \hline
		Permitir al paciente valorar el servicio prestado por una unidad de diagnóstico
	\end{tabularx}

	\bigskip

	\begin{tabularx}{\textwidth}{X}
		\textbf{Resumen}\\ \hline
		El paciente accede al apartado de valoraciones del sistema e indica la unidad de diagnóstico que quiere valorar. El sistema almacena y procesa la valoración.
	\end{tabularx}

	\bigskip

	\begin{tabularx}{\textwidth}{X}
		\textbf{Curso Normal (Básico)}\\ \hline
	\end{tabularx}
	\begin{tabularx}{\textwidth}{cXcX}
		\textbf{1} & Paciente: Accede al servicio web & & \\
		\textbf{2} & Paciente: Accede al apartado de valoraciones & & \\
		\textbf{3} & Paciente: Selecciona la unidad de diagnóstico que quiere valorar. & \textbf{4} & Almacena y procesa la valoración \\
	\end{tabularx}
	
	\begin{tabularx}{\textwidth}{X}
		\textbf{Cursos Alternos}\\ \hline
	\end{tabularx}
\end{table}

\begin{table}[H]
	\begin{tabularx}{\textwidth}{X}
		\textbf{Otros datos}\\ \hline
	\end{tabularx}
	\begin{tabularx}{\textwidth}{lXlX}
		\textbf{Frecuencia esperada} & Muy baja & \textbf{Rendimiento} & Alto\\
		\textbf{Importancia} & Media & \textbf{Urgencia} & Baja\\
		\textbf{Estado} &  & \textbf{Estabilidad} & Alta \\
	\end{tabularx}
	
	\begin{tabularx}{\textwidth}{X}
		\textbf{Comentarios}\\ \hline
	\end{tabularx}
\end{table}

\newpage

% FIN VALORACIÓN REMOTA DE UNIDAD DE DIAGNOSTICO

% GLOSARIO DE TÉRMINOS
\section{Glosario de términos}
\term{Historial clínico} Conjunto de documentos que contienen datos, valoraciones e información sobre la evolución clínica de un paciente a lo largo del proceso asistencial.

\term{Sistema de avisos} Plataforma utilizada para informar en momentos concretos de diversos acontecimientos.

\term{Recurso} Espacio, sala o maquinaria (despachos de consulta, salas de aparatos, apartados de rayos X...).

\term{Cura} Tratamiento que se sigue para curar o aliviar una enfermedad, una herida o un daño físico.

\term{Valoración} Documento en el que se recoge la opinión de un paciente sobre un servicio prestado.



% FIN GLOSARIO DE TÉRMINOS
\end{document}

