\documentclass[11pt,a4paper]{article}

\usepackage[headsep=1cm,headheight=3cm,left=3.5cm,right=3.5cm,top=2.5cm,bottom=2.5cm,a4paper]{geometry}

\linespread{1.3}
\setlength{\parindent}{0pt}
\setlength{\parskip}{1em}

\usepackage[spanish]{babel}
\usepackage[utf8]{inputenc}

%% Fuentes personalizadas para utilizar con XeTeX
\usepackage[sfdefault]{roboto}
\usepackage[scaled=0.9]{DejaVuSansMono}
\usepackage[T1]{fontenc}

\usepackage{enumitem}
\setlist[itemize]{leftmargin=*}
\setlist[enumerate]{leftmargin=*}

\usepackage{changepage}

\newcommand{\term}[2]{\textbf{#1}\quad#2\\}

\newcounter{ActCounter}
\newcommand{\act}[1]{\addtocounter{ActCounter}{1}\textbf{\sffamily ACT-\theActCounter}\quad#1\\}

\newcounter{CUCounter}
\newcommand{\cu}[1]{\addtocounter{CUCounter}{1}\textbf{\sffamily CU-\theCUCounter}\quad#1\\}

\usepackage{tabularx}
\usepackage{float}
\usepackage{adjustbox}

\title{Práctica 2: Modelo de casos de uso \large\\ Fundamentos de Ingeniería del Software}
\author{Sofía Almedia Bruno \and José Antonio Álvarez Ocete \and Miguel Lentisco Ballesteros \and Simón López Vico \and José María Martín Luque}

\begin{document}

\maketitle


\begin{table}[H]
	\begin{tabularx}{\textwidth}{l|Xlllr}
		\textbf{Caso de Uso}   & Recordatorio de citas & & & & \cu \\  
		\textbf{Actores}       & Temporizador & & & \\ 
		\textbf{Tipo}          & Opcional & & & \\
		\textbf{Referencias}   & \multicolumn{5}{>{\hsize=\dimexpr\textwidth-\hsize\relax}X}{RF-6, RN-3}\\
		\textbf{Precondición}  & \multicolumn{5}{>{\hsize=\dimexpr\textwidth-0.85\hsize\relax}X}{Plataforma activa y operativa, servicio de mensajería activo y existencia de cita a recordar}\\ 
		\textbf{Postcondición} & \multicolumn{5}{>{\hsize=\dimexpr\textwidth-0.85\hsize\relax}X}{Recordatorio correctamente enviado.}\\
		\textbf{Autor}         & Grupo ? & \textbf{Fecha} & 07/04/18 & \textbf{Versión} & 1.0 \\ 
	\end{tabularx}

	\bigskip

	\begin{tabularx}{\textwidth}{X}
		\textbf{Propósito}\\ \hline
		Recordar al paciente que tiene una cita. Este acción se puede regular dependiendo del paciente si este quiere que se le notifique o no, con qué regularidad o con cuánta antelación
	\end{tabularx}

	\bigskip

	\begin{tabularx}{\textwidth}{X}
		\textbf{Resumen}\\ \hline
		Se inicia el proceso del temporizador (ya sea de forma periódica o a una hora programada), consulta las citas programadas y envía un recordatorio a aquellos clientes que así lo tengan especificado, cumpliendo restricciones de tiempo especificadas por los mismos
	\end{tabularx}

	\bigskip

	\begin{tabularx}{\textwidth}{X}
		\textbf{Curso Normal (Básico)}\\ \hline
	\end{tabularx}
	\begin{tabularx}{\textwidth}{cXcX}
		\textbf{1} & Actor 1: Acción realizada por el actor & & \\
		\textbf{2} & Actor 1: Acción realizada por el actor & \textbf{3} & Acción realizada por el sistema \\
		 & & & \\
		 & & \textbf{N} &  Cuando se realiza la inclusión de otro caso de uso lo representaremos de la forma Incluir(CU\_identificador.CU\_Nombre)\\
		 & & & \\
		 & << Se incluyen la secuencia de acciones realizadas por los actores que intervienen en el CU, se usarán frases cortas que describan el diálogo entre los actores y el sistema >> << Se pueden añadir referencias a elementos de un boceto de Interfaz del Usuario>> & & << Se incluyen la secuencia de acciones que realiza el sistema ante las acciones de los actores >>
	\end{tabularx}
\end{table}

\begin{table}[H]
	\begin{tabularx}{\textwidth}{X}
		\textbf{Cursos Alternos}\\ \hline
	\end{tabularx}
	\begin{tabularx}{\textwidth}{cX}
		\textbf{1a} & Descripción de la secuencia de acciones alternas a la acción 1 del Curso Normal\\
		\textbf{1b} & \\
		 & << Secuencia de los cursos alternos del CU >>
	\end{tabularx}
	\begin{tabularx}{\textwidth}{X}
		\textbf{Otros datos}\\ \hline
	\end{tabularx}
	\begin{tabularx}{\textwidth}{lXlX}
		\textbf{Frecuencia esperada} & << Numero de veces que se realiza el CU por unidad de tiempo >> & \textbf{Rendimiento} & << Rendimiento esperado de la secuencia de acciones del CU >>\\
		\textbf{Importancia} & << Importancia de este CU en el sistema (vital, alta, moderada, baja) >> & \textbf{Urgencia} & << Urgencia en la realización de este CU, durante el desarrollo (alta, moderada, baja) >>\\
		\textbf{Frecuencia esperada} & << Estado actual del CU en el desarrollo >> & \textbf{Rendimiento} & << estabilidad de los requisitos asociados a este CU (alta, moderada, baja) >>\\
	\end{tabularx}
	\begin{tabularx}{\textwidth}{X}
		\textbf{Comentarios}\\ \hline
		<< Comentarios adicionales sobre este CU >>
	\end{tabularx}
\end{table}

% TERMINAR CONSULTA 

\begin{table}[H]
	\begin{tabularx}{\textwidth}{l|Xlllr}
		\textbf{Caso de Uso}   & Terminar consulta & & & & \cu \\  
		\textbf{Actores}       & Paciente, sanitario & & & \\ 
		\textbf{Tipo}          & Primario, esencial & & & \\
		\textbf{Referencias}   & \multicolumn{5}{>{\hsize=\dimexpr\textwidth-\hsize\relax}X}{Llamar siguiente paciente}\\
		\textbf{Precondición}  & \multicolumn{5}{>{\hsize=\dimexpr\textwidth-0.85\hsize\relax}X}{El sanitario ya a tratado al paciente}\\ 
		\textbf{Postcondición} & \multicolumn{5}{>{\hsize=\dimexpr\textwidth-0.85\hsize\relax}X}{Paciente sale de la consulta}\\
		\textbf{Autor}         & Sofía Almeida & \textbf{Fecha} & 09/04/18 & \textbf{Versión} & 1.0 \\ 
	\end{tabularx}

        \bigskip

	\begin{tabularx}{\textwidth}{X}
		\textbf{Propósito}\\ \hline
                Si el paciente no tiene más molestias, abandonar la consulta
	\end{tabularx}

	\bigskip

	\begin{tabularx}{\textwidth}{X}
		\textbf{Resumen}\\ \hline
		El sanitario pregunta al paciente si desea que traten más síntomas o problemas que éste pueda tener. Si responde de forma negativa se va de la consulta.
	\end{tabularx}

	\bigskip

	\begin{tabularx}{\textwidth}{X}
		\textbf{Curso Normal (Básico)}\\ \hline
	\end{tabularx}
	\begin{tabularx}{\textwidth}{cXcX}
		\textbf{1} & Sanitario: Pregunta al paciente si tiene alguna consulta más & & \\
		\textbf{2} & Paciente: Responde negativamente & \textbf{3} & Se cierra el historial clínico del paciente \\
		 & & \textbf{4} & Incluir (Llamar siguiente paciente) \\
	\end{tabularx}
        \begin{tabularx}{\textwidth}{X}
	  \textbf{Cursos Alternos}\\ \hline
	\end{tabularx}
	\begin{tabularx}{\textwidth}{cX}
	  \textbf{2a} & El paciente responde afirmativamente. El sanitario volvería a explorar al paciente, atendiendo a los nuevos síntomas descritos \\
	\end{tabularx}
\end{table}

\begin{table}[H]
	\begin{tabularx}{\textwidth}{X}
		\textbf{Otros datos}\\ \hline
	\end{tabularx}
	\begin{tabularx}{\textwidth}{lXlX}
		\textbf{Frecuencia esperada} & Cada media hora & \textbf{Rendimiento} & \\
		\textbf{Importancia} & Alta & \textbf{Urgencia} & Alta\\
	\end{tabularx}
	\begin{tabularx}{\textwidth}{X}
		\textbf{Comentarios}\\ \hline	
	\end{tabularx}
\end{table}
% FIN TERMINAR CONSULTA

% LLAMAR SIGUIENTE PACIENTE
\begin{table}[H]
	\begin{tabularx}{\textwidth}{l|Xlllr}
		\textbf{Caso de Uso}   & Llamar siguiente paciente & & & & \cu \\  
		\textbf{Actores}       & Sanitario, sistema de avisos & & & \\ 
		\textbf{Tipo}          & Primario, esencial & & & \\
		\textbf{Referencias}   & \multicolumn{5}{>{\hsize=\dimexpr\textwidth-\hsize\relax}X}{Terminar consulta}\\
		\textbf{Precondición}  & \multicolumn{5}{>{\hsize=\dimexpr\textwidth-0.85\hsize\relax}X}{La consulta ya ha terminado}\\ 
		\textbf{Postcondición} & \multicolumn{5}{>{\hsize=\dimexpr\textwidth-0.85\hsize\relax}X}{Habrá un nuevo aviso en el sistema de avisos}\\
		\textbf{Autor}         & Sofía Almeida & \textbf{Fecha} & 09/04/18 & \textbf{Versión} & 1.0 \\ 
	\end{tabularx}

	\bigskip

	\begin{tabularx}{\textwidth}{X}
		\textbf{Propósito}\\ \hline
                Notificar al siguiente paciente de que ya puede pasar a consulta
	\end{tabularx}

	\bigskip

	\begin{tabularx}{\textwidth}{X}
		\textbf{Resumen}\\ \hline
                El sanitario indica al sistema que quiere llamar a un nuevo paciente, el sistema buscará cuál es y se emitirá un aviso a través del sistema de avisos
        \end{tabularx}

	\bigskip

	\begin{tabularx}{\textwidth}{X}
		\textbf{Curso Normal (Básico)}\\ \hline
	\end{tabularx}
	\begin{tabularx}{\textwidth}{cXcX}
		\textbf{1} & Sanitario: Indica que quiere llamar a un nuevo paciente & & \\
	        & & \textbf{2} & Busca siguiente paciente \\
		& & \textbf{3} & Informa del siguiente paciente \\
		\textbf{4} & Sistema de avisos: Notifica al siguiente paciente & & \\
	\end{tabularx}
	
	\begin{tabularx}{\textwidth}{X}
	  \textbf{Cursos Alternos}\\ \hline
	\end{tabularx}
	\begin{tabularx}{\textwidth}{cX}
		\textbf{2a} & Si no hay más pacientes se espera hasta que sea la hora del siguiente paciente y continúa con el curso normal\\
	\end{tabularx}
\end{table}

\begin{table}[H]
	\begin{tabularx}{\textwidth}{X}
		\textbf{Otros datos}\\ \hline
	\end{tabularx}
	\begin{tabularx}{\textwidth}{lXlX}
		\textbf{Frecuencia esperada} & Cada media hora & \textbf{Rendimiento} & Alto en la mayor parte de los casos\\
		\textbf{Importancia} & Alta & \textbf{Urgencia} & Moderada\\
	\end{tabularx}
	
	\begin{tabularx}{\textwidth}{X}
		\textbf{Comentarios}\\ \hline
		El sistema de avisos puede ser muy diferente (pantalla, altavoz, ...) pero las acciones a realizar son las mismas
	\end{tabularx}
\end{table}

% FIN LLAMAR SIGUIENTE PACIENTE

% GLOSARIO DE TÉRMINOS
\section{Glosario de términos}
\term{Historial clínico} Conjunto de documentos que contienen datos, valoraciones e información sobre la evolución clínica de un paciente a lo largo del proceso asistencial.

\term{Sistema de avisos} Plataforma utilizada para informar en momentos concretos de diversos acontecimientos.

% FIN GLOSARIO DE TÉRMINOS
\end{document}

